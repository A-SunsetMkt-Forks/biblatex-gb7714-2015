
% !Mode:: "TeX:UTF-8"
% 用于测试gb7714-2015样式,实现GB/T 7714-2015 标准说明中给出的顺序编码制示例
\documentclass{article}
\usepackage{ctex}
\usepackage[russian,french,english]{babel}
\usepackage{fontspec}
 \setmainfont{CMU Serif}
 \setCJKmainfont{SourceHanSerifSC-Regular.otf}
\usepackage{xcolor}
\usepackage{toolbox}
\usepackage[colorlinks]{hyperref}
\usepackage{lipsum}
\usepackage[paperwidth=21cm,paperheight=29cm,top=3cm,bottom=2cm,left=1.5cm,right=1.5cm]{geometry}
\usepackage{xltxtra,mflogo,texnames}
%%注意:这里使用other*来使得babel切换环境时忽略空白,使得条目集中各文献间的标点正确
\usepackage[backend=biber,style=gb7714-2015,gbfootbib=true,gbalign=gb7714-2015,autolang=other*]{biblatex}%sorting=nyt



% printbiblist命令需要一个与其参数同名的文献输出驱动,详见92-bibliographylists.tex
\DeclareBibliographyDriver{authorexample}{%
%\usebibmacro{begentry}%
\begingroup
\usebibmacro{author/editor+others/translator+others}\par
%\usebibmacro{finentry}
\endgroup
}


\DeclareFieldFormat{titlea}{#1}

\renewbibmacro*{titlea}{%
  \ifboolexpr{%
    test{\iffieldundef{title}}%
    and
    test{\iffieldundef{subtitle}}%
  }%
    {}%
    {\printtext[titlea]{\bibtitlefont%增加字体控制命令
       \printfield[titlecase]{title}%
       \ifboolexpr{test {\iffieldundef{subtitle}}}%这里增加了对子标题的判断,解决不判断多一个点的问题
       {}{\setunit{\subtitlepunct}%
       \printfield[titlecase]{subtitle}}%
       \iffieldundef{titleaddon}{}%判断一下titleaddon,否则直接加可能多一个空格
        {\setunit{\subtitlepunct}\printfield{titleaddon}}%
}}}

\DeclareBibliographyDriver{titlexamplea}{%
\usebibmacro{begentry}%
\iffieldundef{title}{\usebibmacro{journal}}{\usebibmacro{titlea}}
\usebibmacro{finentry}
}


\DeclareBibliographyDriver{titlexample}{%
\usebibmacro{begentry}%
\iffieldundef{title}{\usebibmacro{journal+issuetitle}}{\usebibmacro{title}}
\usebibmacro{finentry}
}

\DeclareBibliographyDriver{editionexample}{%
%\usebibmacro{begentry}%
\begingroup
\printfield{edition}\par
%\usebibmacro{finentry}
\endgroup
}

\DeclareBibliographyDriver{pubexample}{%
\usebibmacro{begentry}%
\ifentrytype{online}{\usebibmacro{institution+location+date}}{\usebibmacro{publisher+location+date}}%\par
\usebibmacro{chapter+pages}%
\usebibmacro{doi+eprint+url}%从下面移动到上面来,因为gbt2015的url需直接放在页码后面。
\usebibmacro{finentry}
}

\DeclareBibliographyDriver{dateexample}{%
\usebibmacro{begentry}%
\ifentrytype{online}{\usebibmacro{modifydate}}{%
\ifkeyword{news}{\usebibmacro{newsdate}}{\usebibmacro{institution+location+date}}}%\par
\usebibmacro{chapter+pages}%
\usebibmacro{doi+eprint+url}%从下面移动到上面来,因为gbt2015的url需直接放在页码后面。
\usebibmacro{finentry}
}

\DeclareBibliographyDriver{dvnpexample}{%
\usebibmacro{issue+date}%
\iffieldundef{volume}{}{\setunit{\addcomma\space}}%
\usebibmacro{volume+number+eid}%把卷期放到年份后面
\usebibmacro{note+pages}\par%
}

\defbibenvironment{exampleenv}
  {\list
     {示例\printfield{labelnumber}:\space}
     {\setlength{\labelwidth}{1cm}%
      \setlength{\leftmargin}{\labelwidth}%
      \setlength{\labelsep}{\biblabelsep}%
      \addtolength{\leftmargin}{\labelsep}%
      \setlength{\itemsep}{\bibitemsep}%
      \setlength{\parsep}{\bibparsep}%
      \renewcommand*{\makelabel}[1]{##1\hss}}}
  {\endlist}
  {\item}

\defbibheading{biblist}[\bibname]{%
\paragraph*{#1}}

\addbibresource{example.bib}

\begin{document}

{\heiti\large\noindent 本文档用于gb7714-2015更新后与GB/T 7714-2015标准文档的内容进行比较,以确定更新没有引入样式方面的BUG,对比文档为stdgbT7714-2015.pdf,该文档经过多次比较,明确与GB/T 7714-2015中的示例完全一致。

}

\section*{GB/T 7714-2015 中的著录标准和顺序编码制示例}

\section*{4 著录项目和著录格式}

\subsection*{4.1 专著}
\begin{refsection}

\nocite{陈登原2000-29-29,
哈里森沃尔德伦2012-235-236,
北京市政协民族和宗教委员会2012-112-112,
全国信息与文献标准化技术委员会2010-2-3,
徐光宪2010--,
顾炎武1992--,
王夫之1865--,
牛志明2012--,
中国第一历史档案馆2001--,
杨保军2012--,
赵学功2001--,
同济大学土木工程防灾国家重点实验室2011-5-6,
中国造纸学会2003--,
PEEBLES2001--,
Yufin2000--,
Baldock2011-105-105,
Fan2013-25-26
}

\printbibliography[heading=subbibliography,title={示例:}]
\end{refsection}

\subsection*{4.2 专著中的析出文献}
\begin{refsection}

\nocite{
王夫之2011-1109-1109,
程根伟1999-32-36,
陈晋镳1980-56-114a,
马克思2013-302-302,
贾东琴2011-45-52,
Weinstein1974-745-772,
Roberson2011-1-36
}

\printbibliography[heading=subbibliography,title={示例:}]
\end{refsection}

\subsection*{4.3 连续出版物}
\begin{refsection}

\nocite{
中华医学会湖北分会1984----,
中国图书馆学会1957--1990--,
AAAS1883----,
}

\printbibliography[heading=subbibliography,title={示例:}]
\end{refsection}

\subsection*{4.4 连续出版物中的析出文献}
\begin{refsection}

\nocite{袁训来2012-3219-3219,
余建斌2013--,
李炳穆2008-6-12,
李幼平2010-225-228,
武丽丽2008-8-9,
Kanamori1998-2063-2063,
Caplan1993-61-66,
Frese2013-378-398,
Myburg2014-356-362
}

\printbibliography[heading=subbibliography,title={示例:}]
\end{refsection}

\subsection*{4.5 专利文献}
\begin{refsection}

\nocite{邓一刚2006--,
西安电子科技大学2002--,
Tachibana2005--}

{
 \hyphenpenalty=100 %断词阈值,值越大越不容易出现断词
 \tolerance=5000 %丑度,10000为最大无溢出盒子,参考the texbook 第6章
\printbibliography[heading=subbibliography,title={示例:}]
}
\end{refsection}

\subsection*{4.6 电子资源}
\begin{refsection}

\nocite{中国互联网络信息中心2012--,
北京市人民政府办公厅2005--,
Bawden2008--,
OCLC--,
Hopkinson2009--
}

{
 \hyphenpenalty=100 %断词阈值,值越大越不容易出现断词
 \tolerance=5000 %丑度,10000为最大无溢出盒子,参考the texbook 第6章
\printbibliography[heading=subbibliography,title={示例:}]
}
\end{refsection}

\section*{6 著录用文字}
\section*{示例1}
\begin{refsection}
\nocite{周鲁卫2011--}
\nocite{常森2013--}
\nocite{kereanrefa}
\nocite{japaneserefc}
\nocite{RUDDOCK2009--}
\nocite{russianrefc}
\printbibliography[heading=subbibliography,title={用原语种著录参考文献}]
\end{refsection}

\section*{示例2}
\begin{refsection}
\defbibentryset{李炳穆set}{kereanrefb,李炳穆2005--}
\defbibentryset{图书馆信息政策set}{kereanrefc,图书馆信息政策委员会2007--}
\nocite{李炳穆set}
\nocite{图书馆信息政策set}
\printbibliography[heading=subbibliography,title={用韩中2种语言著录文献}]
\end{refsection}

\section*{示例3}
\begin{refsection}
\defbibentryset{熊平set}{熊平2005--,xiong2005--}
\defbibentryset{食品药品监督set}{上海市食品药品监督管理局课题组2008--,Rgsfda2008--}
\nocite{熊平set}
\nocite{食品药品监督set}
\printbibliography[heading=subbibliography,title={用中英2种语言著录文献}]
\end{refsection}


\section*{7 著录用符号}

\begin{refsection}
\nocite{GPS1988--}
\nocite{李约瑟1991--}
\nocite{武丽丽2008-8-9}
\printbibliography[heading=subbibliography,title={符号.示例}]
\end{refsection}

\begin{refsection}
\nocite{WHO1970--}
\nocite{刘彻东1998-38-39}
\nocite{邓一刚2006--}
\printbibliography[heading=subbibliography,title={符号:示例}]
\end{refsection}

\begin{refsection}
\nocite{罗杰斯2011-15-16}
\nocite{袁训来2012-3219-3219}
\printbibliography[heading=subbibliography,title={符号,示例}]
\end{refsection}

\begin{refsection}
\nocite{编者1964--}
\printbibliography[heading=subbibliography,title={符号;示例}]
\end{refsection}

\begin{refsection}
\nocite{李约瑟1991--}
\nocite{张忠智1997-33-34}
\printbibliography[heading=subbibliography,title={符号//示例}]
\end{refsection}

\begin{refsection}
\nocite{DUBAR2013--}
\nocite{余建斌2013--}
\nocite{Bawden2008--}
\printbibliography[heading=subbibliography,title={符号()示例}]
\end{refsection}

\begin{refsection}
\nocite{DUBAR2013--}
\printbibliography[heading=subbibliography,title={符号{[}{]}示例}]
\end{refsection}

\begin{refsection}
\nocite{Dowler1995-5-26}
\nocite{DUBAR2013--}
\printbibliography[heading=subbibliography,title={符号/示例}]
\end{refsection}

\begin{refsection}
\nocite{Dowler1995-5-26}
\nocite{DUBAR2013--}
\printbibliography[heading=subbibliography,title={符号-示例}]
\end{refsection}


\section*{8.1 责任者}
\subsection*{8.1.1 个人著者}
\begin{refsection}
注意其中“昂温”文献的情况,姓名是中英文混合的情况是比较特殊的,因为存在中文,所以判定为中文,但为了输出的形式保持英文中姓和名之间的空格,因此bib文件中的输入以\{\}进行保护。
\nocite{李时珍--}
\nocite{乔纳斯--}
\nocite{昂温1988--}
\nocite{GPS1988--}
\nocite{丸山敏秋--}
\nocite{凯西尔--}
\nocite{Einstein--}
\nocite{Williams-ellis--}
\nocite{morgan--}
\nocite{lijianning--a}
\nocite{lijianning--b}
\printbiblist[title={作者姓名示例},env=exampleenv]{authorexample}
\end{refsection}

\subsection*{8.1.2 多位著者}
\begin{refsection}
\nocite{钱学森--}
\nocite{李四光--}
\nocite{印森林--}
\nocite{fordham--}
\printbiblist[title={作者姓名示例},env=exampleenv]{authorexample}
\end{refsection}

\subsection*{8.1.3 缺省著者}
\begin{refsection}
\nocite{anon1981-628}
\printbibliography[heading=subbibliography,title={作者姓名示例}]
\end{refsection}

\subsection*{8.1.4 团体著者}
\begin{refsection}
\nocite{中国科学院物理研究所--}
\nocite{贵州省土穰普查办公室--}
\nocite{AmericanChemicalSociety--}
\nocite{StanfordUniversity--}
\printbiblist[title={作者姓名示例},env=exampleenv]{authorexample}
\end{refsection}

\section*{8.2 题名}
\begin{refsection}
\nocite{王夫之的诠释--}
\nocite{张子正蒙注--}
\nocite{化学动力学和反应器原理--}
\nocite{袖珍神学--}
\nocite{北京师范大学学报--}
\nocite{Gasesinsea--}
\nocite{jmath--}
\printbiblist[title={题名示例},env=exampleenv]{titlexamplea}
\end{refsection}

\subsection*{8.2.1 合订题名}
\begin{refsection}
\nocite{为人民服务--}
\nocite{大趋势--}
\printbiblist[title={题名示例},env=exampleenv]{titlexamplea}
\end{refsection}


\subsection*{8.2.3/8.2.4 标识符和其它题名信息}
\begin{refsection}
\nocite{地壳运动--}
\nocite{三松堂--}
\nocite{世界出版业--}
\nocite{ECL集成电路--}
\nocite{中国科学技术史--}
\nocite{商鞅战秋菊--}
\nocite{中国科学--}
\nocite{信息与文献--}
\nocite{中子反射--}
\nocite{AsianPacificjournal--}
\printbiblist[title={题名示例},env=exampleenv]{titlexample}
\end{refsection}

\section*{8.3 版本}
\begin{refsection}
\nocite{egbookeda--}
\nocite{egbookedb--}
\nocite{egbookedc--}
\nocite{egbookedd--}
\nocite{egbookede--}
\printbiblist[title={版本示例},env=exampleenv]{editionexample}
\end{refsection}

\section*{8.4 出版项}
\begin{refsection}
\nocite{egbookpuba--}
\nocite{egbookpubb--}
\printbiblist[title={出版项示例},env=exampleenv]{pubexample}
\end{refsection}

\subsection*{8.4.1 出版地}
\begin{refsection}
\nocite{egbookpubaddressa--}
\nocite{egbookpubaddressb--}
\nocite{egbookpubaddressc--}
\nocite{egbookpubaddressd--}
\nocite{egbookpubaddresse--}
\nocite{egbookpubaddressf--}
\nocite{egbookpubaddressg--}
\printbiblist[title={出版地示例},env=exampleenv]{pubexample}
\end{refsection}


\subsection*{8.4.2 出版者}
\begin{refsection}
\nocite{egbookpubpublishera--}
\nocite{egbookpubpublisherb--}
\nocite{egbookpubpublisherc--}
\nocite{egbookpubpublisherd--}
\nocite{egbookpubpublishere--}
\nocite{egbookpubpublisherf--}
\printbiblist[title={出版者示例},env=exampleenv]{pubexample}
\end{refsection}

\subsection*{8.4.3 出版日期}
\begin{refsection}
\nocite{egbookpubdatea--}
\nocite{egbookpubdateb--}
\nocite{egbookpubdatec--}
\nocite{egbookpubdated--}
\nocite{egbookpubdatee--}
\nocite{egbookpubdatef--}
\printbiblist[title={出版日期示例},env=exampleenv]{dateexample}
\end{refsection}

\subsection*{8.4.4 公告更新引用日期}
\begin{refsection}
\nocite{egbookpubdateg--}
\printbiblist[title={公告更新引用日期示例},env=exampleenv]{dateexample}
\end{refsection}

\section*{8.5 页码}
\begin{refsection}
\nocite{曹凌2011-19-}
\nocite{钱学森2001--}
\nocite{冯友兰2008--}
\nocite{李约瑟1991--}
\nocite{DUBAR2013--}
\printbibliography[heading=subbibliography,title=页码示例]
\end{refsection}


\subsection*{8.6 获取和访问路径}
\begin{refsection}

\nocite{储大同2010-721-724,weiner2010-38}

\printbibliography[heading=subbibliography,title={示例}]
\end{refsection}

\subsection*{8.7 数字对象唯一识别符}
\begin{refsection}

\nocite{刘乃安2000-17-18,Deverell2013-21-22}
\printbibliography[heading=subbibliography,title={示例}]
\end{refsection}

\subsection*{8.8 析出文献}

\subsection*{8.8.1 析出关系表示}
\begin{refsection}

\nocite{姚中秋2009--,关立哲2014--,TENOPIR1987--}

\printbibliography[heading=subbibliography,title={示例}]
\end{refsection}

\subsection*{8.8.2-5 年卷期页码}
\begin{refsection}

\nocite{egdatevolnumpagea--,egdatevolnumpageb--,%
egdatevolnumpagec--,egdatevolnumpaged--,%
egdatevolnumpagee--,egdatevolnumpagef--,%
egdatevolnumpageg--}

\printbiblist[title={年卷期页码},env=exampleenv]{dvnpexample}
\end{refsection}

\subsection*{9.2 文献表}
\begin{refsection}

\nocite{Baker1995--,Chernik1982--,尼葛洛庞帝1996--,汪冰1997-16-16,杨宗英1996-24-29,Dowler1995-5-26}

\printbibliography[heading=subbibliography,title={示例}]
\end{refsection}

\subsection*{10.1.1 一处引用一篇文献}

\begin{refsection}
示例1:

所谓移情,就是“说话人将自己认同于......他用句子所描述的时间或状态中的一个参与者”\cite{Sunstein1996-903-903}。《汉语大词典》和张相
\cite{Morri2010--}都认为“可”是“痊愈”,
候精一认为是“减轻”\cite{罗杰斯2011-15-16}。......另外,根据候精一,表示病痛程度减轻的形容词“可”和表示逆转否定的副词“可”
是兼类词\cite{陈登原2000-29-29},这也说明二者应该存在着源流关系。

示例2:

所谓移情,就是“说话人将自己认同于......他用句子所描述的时间或状态中的一个参与者”\footfullcite{Sunstein1996-903-903}。《汉语大词典》和张相
\footfullcite{Morri2010--}都认为“可”是“痊愈”,
候精一认为是“减轻”\footfullcite{罗杰斯2011-15-16}。......另外,根据候精一,表示病痛程度减轻的形容词“可”和表示逆转否定的副词“可”
是兼类词\footfullcite{陈登原2000-29-29},这也说明二者应该存在着源流关系。


\end{refsection}

\subsection*{10.1.2 一处引用多篇文献}
\begin{refsection}

裴伟提出\cite{Humphrey1971--,CRANE1972--}......

莫拉德对稳定区的研究
\cite{CRANE1972--,Weinstein1974-745-772,KENNEDY1975-311-386}......


\end{refsection}


\newpage
\subsection*{10.1.3 多次引用同一著者的同一文献}
\begin{refsection}
示例1:多次引用同一著者的同一文献的序号

……改变社会规范也可能存在类似的“二阶囚徒困境”问题:尽管改变旧的规范对所有人都好,
但个人理性选择使得没有人愿意率先违反旧的规范\cite{Sunstein1996-903-903}。
……事实上,古希腊对轴心时代思想真正的贡献不是来自对民主的赞扬,而是来自对民主制度的批评,苏格拉底、柏拉图和亚里士多德3位贤圣
都是民主制度的坚决反对者\pagescite[260]{Morri2010--}。
……柏拉图在西方世界的影响力是如此之大以至于有学者评论说,一切后世的思想都是一系列为柏拉图思想所作的脚注\cite{罗杰斯2011-15-16}。
……据《唐会要》记载,当时拆毁的寺院有4 600余所,招提、兰若等佛教建筑4万余所,没收寺产,并强迫僧尼还俗达260 500人。
佛教受到极大的打击\pagescite[326-329]{Morri2010--}。
……陈登原先生的考证是非常精确的,他印证了《春秋说题辞》“黍者绪也,故其立字,禾入米为黍,为酒以扶老,为酒以序尊卑,禾为柔物,亦宜养老”,指出:“以上谓等威之辨,尊卑之序,由于饮食荣辱。”\cite{陈登原2000-29-29}

\printbibliography[heading=subbibliography,title={参考文献}]
\end{refsection}

\begin{refsection}
示例2:多次引用同一著者的同一文献的脚注序号

……改变社会规范也可能存在类似的“二阶囚徒困境”问题:尽管改变旧的规范对所有人都好,但个人理性选择使得没有人愿意率先违反旧的规范
\footfullcite{Sunstein1996-903-903}。
……事实上,古希腊对轴心时代思想真正的贡献不是来自对民主的赞扬,而是来自对民主制度的批评,苏格拉底、柏拉图和亚里士多德3位贤圣
都是民主制度的坚决反对者\footfullcite[260]{Morri2010--}。
……柏拉图在西方世界的影响力是如此之大以至于有学者评论说,一切后世的思想都是一系列为柏拉图思想所作的脚注\footfullcite{罗杰斯2011-15-16}。
……据《唐会要》记载,当时拆毁的寺院有4 600余所,招提、兰若等佛教建筑4万余所,没收寺产,并强迫僧尼还俗达260 500人。
佛教受到极大的打击\footfullcite[326-329]{Morri2010--}。
……陈登原先生的考证是非常精确的,他印证了《春秋说题辞》“黍者绪也,故其立字,禾入米为黍,为酒以扶老,为酒以序尊卑,禾为柔物,亦宜养老”,指出:“以上谓等威之辨,尊卑之序,由于饮食荣辱。”\footfullcite{陈登原2000-29-29}
\end{refsection}

\end{document}
