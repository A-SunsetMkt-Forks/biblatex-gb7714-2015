
% !Mode:: "TeX:UTF-8"
% 用于测试gb7714-2015样式,实现GB/T 7714-2015 标准说明中给出的著者年份制示例
\documentclass{article}
\usepackage{ctex}
\usepackage{xcolor}
\usepackage{toolbox}
\usepackage[colorlinks]{hyperref}
\usepackage{lipsum}
\usepackage[paperwidth=21cm,paperheight=29cm,top=3cm,bottom=2cm,left=1.5cm,right=1.5cm]{geometry}
\usepackage{xltxtra,mflogo,texnames}
\usepackage[backend=biber,style=gb7714-2015ay,gbnoauthor,
%,sortlocale=zh__stroke,nosortothers=true
]{biblatex}%sorting=nyt

%\usepackage{filecontents}
%\begin{filecontents}{\jobname.bib}
%
%\end{filecontents}

\addbibresource{example.bib}
%\renewcommand{\thefootnote}{\textcircled{\tiny\arabic{footnote}}}


\begin{document}


\setcounter{section}{8}
\setcounter{subsection}{1}
\setcounter{subsubsection}{2}
\subsubsection{缺省著者}
\begin{refsection}
\nocite{anon1981-628}
\printbibliography[heading=subbibliography]
\end{refsection}

\stepcounter{section}
\setcounter{subsection}{1}
\subsection{文献表}
\begin{refsection}

\nocite{尼葛洛庞帝1996--,汪冰1997-16-16,杨宗英1996-24-29,Baker1995--,Chernik1982--,Dowler1995-5-26}

\printbibliography[heading=subbibliography]
\end{refsection}

\stepcounter{section}
\setcounter{subsection}{1}
\subsection{著者年份制}
\subsubsection{引用单篇文献}
\begin{refsection}

the notion of an invisible college has been explored in thesciences\cite{CRANE1972--}.Its absence among historians was noted by Stieg\yearcite{STIEG1981-549-560} ...

\printbibliography[heading=subbibliography]
\end{refsection}



\subsubsection{多著者、同著者同年份多篇文献}
\begin{refsection}

\nocite{王临慧2010-147,王临慧2010-138}
\nocite{KENNEDY1975-311-386,KENNEDY1975-339-360}


\printbibliography[heading=subbibliography]

注意:由于biblatex在进行姓名列表非歧义处理时,通常会将作者列表截短后形成的etal考虑在内,所以王临慧这两篇文献的作者列表认为是不同的。但biblatex3.13(biber2.13)引入了一个新的选项,用于在非歧义处理中忽略etal,所以会将两篇文献的列表认为相同,所以会添加a,b以区分。
\end{refsection}

\stepcounter{subsubsection}
\subsubsection{多次引用同一著者的同一文献}
\begin{refsection}
主编靠编辑思想指挥全局已是编辑界的共识\cite{张忠智1997-33-34},然而对编辑思想至今没有一个明确的界定,故不妨提出一个构架……参与讨论。
由于“思想”的内涵是“客观存在反映在人的意识中经过思维活动而产生的结果”
\pagescite[1194]{中国社会科学院语言研究所词典编辑室1996--},所以“编辑思想”的内涵就是编辑实践反映在编辑工作者
的意识中,“经过思维活动而产生的结果”。
……《中国青年》杂志创办人追求的高格调-
理性的成熟与热点的凝聚\cite{刘彻东1998-38-39},表明其读者群的文化的品位的高层次……“方针”指“引导事业前进的方向和目标”
\pagescite[235]{中国社会科学院语言研究所词典编辑室1996--}。
……对编辑方针,1981年中国科协副主席裴丽生曾有过科学的论断—“自然科学学术期刊必须坚持以马列主义、毛泽东思想为指导,贯彻为国民经济发展服务,理论与实践相结合,普及与提高相结合,‘百花齐放,百家争鸣’的方针。” \cite{裴丽生1981-2-10}它完整地回答了为谁服务怎样服务,如何服务得更好的间题。

\printbibliography[heading=subbibliography]
\end{refsection}

\end{document}
