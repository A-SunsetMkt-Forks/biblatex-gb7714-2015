\documentclass{article}
    \usepackage{ctex}
    \usepackage{xcolor}
    \usepackage{toolbox}
    \usepackage[colorlinks]{hyperref}
    \usepackage{lipsum}
    \usepackage[top=3cm,bottom=3cm,left=3cm,right=3cm]{geometry}

\usepackage[backend=biber,style=chinese-jmw,gbvolnum=false]{biblatex}

\begin{filecontents}[force]{\jobname.bib}
@article{Weiss2010,
	title = {Cost behavior and analysts’ earnings forecasts},
	volume = {85},
	doi = {10.2308/accr.2010.85.4.1441},
	journal = {The Accounting Review},
	author = {Weiss, D.},
	year = {2010},
	pages = {1441--1471},
}

@article{Banker2013,
	title = {Employment protection legislation, adjustment costs and cross-country differences in cost behavior},
	volume = {55},
	issn = {0165-4101},
	url = {https://www.sciencedirect.com/science/article/pii/S0165410112000614},
	doi = {10/ghvxsx},
	number = {1},
	urldate = {2022-11-17},
	journal = {Journal of Accounting and Economics},
	author = {Banker, Rajiv D. and Byzalov, Dmitri and Chen, Lei Tony},
	year = {2013},
	pages = {111--127},
}

@article{王化成2015,
	title = {监督还是掏空:大股东持股比列与股价崩盘风险},
	number = {2},
	journal = {管理世界},
	author = {王化成 and 曹丰 and 叶康涛},
	year = {2015},
}

@article{戴治勇2014,
	title = {法治、信任与企业激励薪酬设计},
	issn = {1002-5502},
	url = {https://kns.cnki.net/kcms/detail/frame/list.aspx?dbcode=CJFD&filename=glsj201402011&dbname=CJFDLAST2015&RefType=1&vl=utztcWkJCfgpXOdHpPRrTzeJ4gZY4knPFX_azlr1TYX-KcEl_kKZe5fQaLHjSi8N},
	doi = {10/grvn5k},
	number = {2},
	urldate = {2022-11-17},
	journal = {管理世界},
	author = {戴治勇},
	year = {2014},
	pages = {102--110},
}

@book{李实2004,
	title = {经济转型的代价──中国城市失业、贫困、收入差距的经验分析},
	publisher = {北京中国财政经济出版社},
	editor = {李实 and 佐藤宏},
	year = {2004},
}

@article{林乐2017,
	title = {分析师荐股更新利用管理层语调吗?——基于业绩说明会的文本分析},
	shorttitle = {分析师荐股更新利用管理层语调吗?},
	number = {11},
	journal = {管理世界},
	author = {林乐 and 谢德仁},
	year = {2017},
	pages = {125--145},
}

@incollection{Fama1989,
	title = {Perspectives on {October} 1987, or what did we learn from the crash?},
	language = {en},
	booktitle = {Black monday and the future of the financial markets},
	publisher = {Irwin, Homewood, III},
	author = {Fama, Eugene Francis},
	editor = {Barro, Robert J. and Kamphuis, Robert W. and Kormendi, Roger C. and Watson, J. W. Henry},
    editortype={editor},
	year = {1989},
}

@article{Kang2008,
	title = {The geography of block acquisitions},
	volume = {63},
	doi = {10.1111/j.1540-6261.2008.01414.x},
	number = {6},
	journal = {The Journal of Finance},
	author = {Kang, Jun Koo and Kim, Jin Mo},
	year = {2008},
	pages = {2817--2858},
}

@book{Skolnik1990,
	address = {New York},
	title = {Radar handbook},
	publisher = {McGraw-Hill},
	author = {Skolnik, Merrill I.},
	year = {1990},
}

@misc{Krugman2006,
	type = {{NBER} {Working} {Paper}},
	title = {Title of the article},
	author = {Krugman, P.},
    number={4567},
	year = {2006},
}

@article{Krugman2006a,
	journal = {{NBER} {Working} {Paper}},
	title = {Title of the article},
	author = {Krugman, P.},
    number={4567},
	year = {2006},
}


@article{Krugman2006b,
    entrysubtype={workpaper},
	journal = {{NBER} {Working} {Paper}},
	title = {Title of the article},
	author = {Krugman, P.},
    number={4567},
	year = {2006},
}


@phdthesis{黄超2017,
	type = {博士学位论文},
	title = {管理层利用语调管理配合盈余管理了吗?——来自我国上市公司年报的文本分析},
	school = {上海财经大学},
	author = {黄超},
	year = {2017},
}

@book{高琳2016,
	title = {分税制、地方财政自主权与经济发展绩效研究},
	url = {https://kns.cnki.net/kcms/detail/detail.aspx?dbcode=CDFD&dbname=CDFDLAST2015&filename=1014448470.nh&uniplatform=NZKPT&v=-UqoEpfWi8NUeI8GEqkOHQVdQjdObNdVrbdyS4fiwlOthpQaAioRCwpg1qpW5m3v},
	urldate = {2022-11-17},
	publisher = {上海人民出版社},
	author = {高琳},
	year = {2016},
}

@incollection{佐藤宏2004,
	address = {北京},
	title = {外出务工、谋职和城市劳动力市场——市场支撑机制的社会网络分析},
	booktitle = {经济转型的代价──中国城市失业、贫困、收入差距的经验分析},
	publisher = {中国财政经济出版社},
	author = {佐藤宏},
	editor = {李实 and 佐藤宏},
	year = {2004},
}



\end{filecontents}
    \addbibresource{\jobname.bib}
    %

%注意编者的类型:
%editor:主编
%compiler:整理
%redactor:校订
%reviser:修订
%founder:创建
%continuator:继承者
%collaborator:合作者

\begin{document}

\section{引用标注格式}

\subsection{中文}
正文中中文参考文献的格式:

(1)当作者为一个人时:作者姓名(2018)或者(作者姓名,2018)

\citet{戴治勇2014}  或者 \cite{戴治勇2014}

(2)作者为两个人时:作者1和作者2(2018)或者(作者1、作者2,2018)

\citet{林乐2017}  或者 \cite{林乐2017}

(3)作者为三个及以上时: 第一作者\ 等(2018)或者(第一作者等,2018)

\citet{王化成2015}  或者 \cite{王化成2015}


(4)当引用多个作者时: (第一作者等,2018;作者1、作者2,2018;作者姓名,2018)

\cite{高琳2016,佐藤宏2004,黄超2017}

\cite{戴治勇2014,林乐2017,黄超2017}

\subsection{英文}

正文中英文参考文献的格式:

(1)当作者为一个人时:Betts(2018)或者(Betts,2018)

\citet{Weiss2010}  或者 \cite{Weiss2010}

(2)当作者为两个人时:Betts和Taylor(2018)或者(Betts and Taylor,2018)

\citet{Kang2008}  或者 \cite{Kang2008}

(3)当作者为三个及以上时:Betts等(2018)或者(Betts et al.,2018)

\citet{Banker2013}  或者 \cite{Banker2013}


(4)当引用多个作者时:(Betts et al.,2018;Betts and Taylor,2018;Betts,2018)

\cite{Weiss2010,Kang2008,Krugman2006}

\cite{Fama1989,Skolnik1990}



\section{文后参考文献著录格式}

\subsection{中文}
(1)一个中文作者的参考文献的格式:

戴治勇:《法治、信任与企业激励薪酬设计》,《管理世界》,2014年第2期。
\footfullcite{戴治勇2014}

(2)两个中文作者的参考文献的格式:

林乐、谢德仁:《分析师荐股更新利用管理层语调吗?——基于业绩说明会的文本分析》,《管
理世界》,2017年第11期。
\footfullcite{林乐2017}

(3)三个中文作者的参考文献的格式:

王化成、曹丰、叶康涛:《监督还是掏空:大股东持股比列与股价崩盘风险》,《管理世
界》,2015年第2期。
\footfullcite{王化成2015}

(4)参考文献为中文专著时的格式:

高琳:《分税制、地方财政自主权和经济发展绩效研究》,上海人民出版社,2016年。
\footfullcite{高琳2016}

(5)参考文献为中文论文集中的一篇论文时的格式:

佐藤宏:《外出务工、谋职和城市劳动力市场——市场支撑机制的社会网络分析》,载李实、佐
藤宏主编,《经济转型的代价──中国城市失业、贫困、收入差距的经验分析》,中国财政经济出版
社,2004年。
\footfullcite{佐藤宏2004}

(6)参考文献为毕业论文的格式:

黄超:《管理层利用语调管理配合盈余管理了吗?——来自我国上市公司年报的文本分析》,上
海财经大学博士学位论文,2017年。
\footfullcite{黄超2017}

\subsection{英文}


作者姓写在前,名写在后缩写为首字母;杂志名、书名需斜体;如同
一作者在同一年刊登两篇以上文章,请在年份后区分a,b,……。)

(1)一个英文作者的参考文献的格式:

Weiss,D.,2010,“Cost Behavior and Analysts’ Earnings Forecasts”,\textbf{\textit{The
Accounting Review}},vol.85,pp.1441\verb|~|1471.
\footfullcite{Weiss2010}

(2)两个英文作者的参考文献的格式:

Kang,J. K. and Kim,J. M.,2008,“The Geography of Block Acquisitions”,\textbf{\textit{Journal
of Finance}},63(6),pp.2817\textasciitilde 2858.
\footfullcite{Kang2008}


(3)两个以上英文作者的参考文献的格式:

Banker,R. D.,Byzalov,D. and Chen,L. T.,2013,“Employment Protection
Legislation,Adjustment Costs and Cross-country Differences in Cost Behavior”,\textbf{\textit{Journal
of Accounting and Economics}},vol.55,pp.111$\sim$127.
\footfullcite{Banker2013}

(4)引用文献是工作论文的参考文献格式:

Krugman,P.,2006,“Title of the Article”,NBER Working Paper,No.4567.
\footfullcite{Krugman2006}
\footfullcite{Krugman2006a}
\footfullcite{Krugman2006b}


(5)引用文献是论文集或合集中的某一篇论文的参考文献格式:

Fama,E. F.,1989,“Perspectives on October 1987,or What did We Learn from the
Crash?”,in Barro,R. J.,R. W. Kamphuis,R. C. Kormendi and J. W. H. Watson,eds:
\textbf{\textit{Black Monday and the Future of the Financial Markets}},Irwin,Homewood,III.
\footfullcite{Fama1989}

(6)引用文献是专著时的参考文献格式:

Skolnik,M. I.,1990,\textbf{\textit{Radar handbook}},New York:McGraw-Hill.
\footfullcite{Skolnik1990}

\newpage
{
%\hyphenation{Proce-edings}
\hyphenpenalty=100 %断词阈值, 值越大越不容易出现断词
\tolerance=5000 %丑度, 10000为最大无溢出盒子, 参考the texbook 第6章
\hbadness=100 %如果丑度超过hbadness这一阀值, 那么就会发出警告
    \printbibliography

}

    \end{document} 