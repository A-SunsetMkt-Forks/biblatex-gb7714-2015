\documentclass{article}
\usepackage{ctex}
\usepackage{xcolor}
\usepackage[margin=2cm]{geometry}
\usepackage[colorlinks]{hyperref}
\usepackage{xurl}
\usepackage[backend=biber,style=chinese-cajhss,gbfootbib=true]{biblatex}
\newcommand{\qd}[1]{\textbf{\textcolor{blue}{#1}}}

\begin{filecontents}[force]{\jobname.bib}

@book{戴裔煊1984,
	title = {〈明史·佛郎机传〉笺正},
	pages = {6},
	address={北京},
	publisher = {中国社会科学出版社},
	author = {戴裔煊},
	date = {1984},
	note = {Google-Books-{ID}: gcj\_OAAACAAJ},
}

@book{许毅1996,
	title = {清代外债史论},
	isbn = {978-7-5005-2983-5},
	pages = {95},
	address={北京},
	publisher = {中国财政经济出版社},
	author = {许毅},
	date = {1996},
	langid = {pinyin},
	note = {Google-Books-{ID}: {hC}1tAAAAIAAJ},
}

@book{马士1991,
	title = {东印度公司对华贸易编年史:1635-1834},
	isbn = {978-7-306-00212-9},
	shorttitle = {东印度公司对华贸易编年史},
	pages = {429--431},
	address={广州},
	publisher = {中山大学出版社},
	author = {马士},
	year = {1991年,第 4、5 卷合订本},
	note = {Google-Books-{ID}: {CZVGYAAACAAJ}},
}


@incollection{汪子春1978,
	location = {北京},
	title = {中国养蚕科学技术的发展和传播},
	pages = {382--391},
	booktitle = {中国古代科技成就},
	publisher = {中国青年出版社},
	author = {汪子春},
	editor = {自然科学史研究所},
	date = {1978},
	note = {Google-Books-{ID}: {jfkZy}3zZ57QC},
}


@incollection{鲁迅2005,
	location = {北京},
	title = {中国小说的历史的变迁},
	isbn = {978-7-02-005033-8},
	pages = {325},
	booktitle = {鲁迅全集},
	number={9},
	publisher = {人民文学出版社},
	author = {鲁迅},
	date = {2005},
	note = {Google-Books-{ID}: {ZYVpAAAAIAAJ}},
}

@incollection{唐振常1997,
	title = {师承与变法},
	isbn = {978-7-5325-2188-3},
	booktitle = {识史集},
	address={上海},
	publisher = {上海古籍出版社},
	author = {唐振常},
	date = {1997},
	pages = {65},
}


@incollection{李志刚1989,
	title = {早期传教士在港创办的第一份中文报刊――〈遐迩贯珍〉},
	isbn = {978-7-208-01896-9},
	pages = {135--143},
	booktitle = {基督教与近代文化},
	location = {台北},
	publisher = {宇宙光出版社},
	author = {李志刚},
	date = {1989},
	note = {Google-Books-{ID}: {HWxnAAAAIAAJ}},
}


@inbook{蔡元培2000,
	title = {复孙毓修函},
	entrysubtype = {letter},
	author = {蔡元培},
	editor = {高平叔 and 王世儒},
	editortype = {编注},
	Volume= {上册},
	BookTitle= {蔡元培书信集},
	date = {2000},
	location = {杭州},
	publisher = {浙江教育出版社},
	eventdate={1911-06-03},
	Pages={99},
	note = {Google-Books-{ID}: F95jAAAAIAAJ},
}


@incollection{黄仁宇1997,
	location = {北京},
	title = {为什么称为“中国大历史”?——中文版自序},
	shorttitle = {为什么称为“中国大历史”?},
	pages = {2},
	booktitle = {中国大历史},
	publisher = {三联书店},
	author = {黄仁宇},
	date = {1997},
}

@incollection{楼适夷1988,
	location = {北京},
	title = {读家书,想傅雷(代序)},
	pages = {2},
	booktitle = {傅雷家书},
	volume = {(增补本)},
	publisher = {三联书店},
	author = {楼适夷},
	editor = {傅敏},
	date = {1988},
}


@book{姚际恒光绪三,
	entrysubtype={classic},
	edition = {光绪三年苏州文学山房活字本},
	title = {古今伪书考},
	volume = {卷三},
	author = {姚际恒},
	pages={第九页a},
}


@book{清史稿1977,
	location = {北京},
	title = {清史稿},
	volume = {卷四八六《文苑三·吴汝纶传》},
	pages = {13444},
	publisher = {中华书局},
	year = {1977年标点本},
}

@book{屈大均1985,
	location = {北京},
	title = {广东新语},
	volume = {卷九《东莞城隍》},
	pages = {215},
	publisher = {中华书局},
	author = {屈大均},
	date = {1985年标点本},
}


@book{杨钟羲1991,
	location = {沈阳},
	title = {雪桥诗话续集},
	volume = {卷五},
	publisher = {辽沈书社},
	author = {杨钟羲},
	year = {1991年影印本,上册},
	pages={第461页下栏}
}

@book{太平御览1985,
	location = {北京},
	title = {太平御览},
	volume = {卷六九《服章部七》引《魏台访议》},
	publisher = {中华书局},
	year = {1985年影印本,第3册},
	pages={第3080页下栏}
}


@book{资治通鉴,
	location = {北京},
	title = {资治通鉴},
	volume = {卷二,唐高宗永徽六年十月乙卯},
	publisher = {中华书局},
	year = {1956年标点本,第?册},
	pages={6293}
}

@book{嘉定县志,
	entrysubtype={classic},
	origyear={乾隆},
	title = {嘉定县志},
	volume = {卷十二《风俗》},
	pages={第七页a}
}

@incollection{广东通志,
	entrysubtype={classic},
	location = {北京},
	edition = {影印本},
	title = {广东通志},
	part = {卷十五《郡县志二·广州府·城池》},
	booktitle = {稀见中国地方志汇刊},
	number={42},
	publisher = {中国书店},
	date = {1992},
	origyear={万历},
	pages={367}
}

@incollection{管志道1997,
	entrysubtype={classic},
	location = {济南},
	edition = {影印本},
	title = {答屠仪部赤水丈书},
	volume = {卷二,《四库全书存目丛书·子部》第88册},
	booktitle = {续问辨牍},
	publisher = {齐鲁书社},
	author = {管志道},
	date = {1997},
	pages={73}
}

@incollection{何承矩1987,
	entrysubtype={classic},
	location = {上海},
	title = {上太宗论塘泊屯田之利},
	volume = {卷一〇五,《文渊阁四库丛书》第432册},
	pages = {284},
	booktitle = {宋名臣奏议},
	publisher = {上海古籍出版社},
	author = {何承矩},
	editor = {赵汝愚},
	date = {1987},
}


@book{庄子天下,
title={庄子·天下},
}

@book{尚书泰誓,
title={尚书·泰誓},
}

@book{尚书皋陶谟,
title={尚书·皋陶谟},
}



@article{吴艳红2000,
	title = {明代流刑考},
	number = {6},
	journaltitle = {历史研究},
	author = {吴艳红},
	date = {2000},
}

@article{吴虞1917,
	title = {家族制度为专制主义之根据论},
	issue = {第 2 卷第 6 号},
	journaltitle = {新青年},
	author = {吴虞},
	date = {1917-02-01},
}

@article{王晴佳1998,
	title = {中国二十世纪史学与西方--论现代历史意识的产生},
	issue = {(台北)第 9 卷第 1 期},
	pages = {55--82},
	author = {王晴佳},
	date = {1998-03},
	journaltitle = {新史学},
}


@article{魏丽英1990,
	title = {论近代西北人口波动的主要原因},
	number = {6},
	journaltitle = {社会科学},
	author = {魏丽英},
	date = {1990},
	location={兰州}
}

@article{费成康1999,
	title = {葡萄牙人如何进入澳门问题辨证},
	number = {9},
	journaltitle = {社会科学},
	author = {费成康},
	date = {1999},
	location={上海},
}

@article{董一沙2001,
	title = {回忆父亲董希文},
	number = {3},
	journaltitle = {传记文学},
	author = {董一沙},
	date = {2001},
	location={北京},
}



@article{李济1976,
	title = {创办史语所与支持安阳考古工作的贡献},
	issue={第28卷第1期},
	journaltitle = {传记文学},
	author = {李济},
	date = {1976-01},
	location= {台北},
}


@article{黄义豪1997,
	title = {评黄龟年四劾秦桧},
	number = {3},
	journaltitle = {福建论坛},
	author = {黄义豪},
	date = {1997},
	series = {文史哲版},
}

@article{苏振芳1996,
	title = {新加坡推行儒家伦理道德教育的社会学思考},
	number = {3},
	journaltitle = {福建论坛},
	author = {苏振芳},
	date = {1996},
	series = {经济社会版},
}

@article{叶明勇2001,
	title = {英国议会圈地及其影响},
	number = {2},
	journaltitle = {武汉大学学报},
	author = {叶明勇},
	date = {2001},
	series = {人文科学版},
}


@article{倪素香2002,
	title = {德育学科的比较研究与理论探索},
	pages = {512--513},
	number = {4},
	journaltitle = {武汉大学学报},
	author = {倪素香},
	date = {2002},
	series = {社会科学版},
}


@newspaper{李眉1986,
	title = {李劼人轶事},
	pages = {2},
	journaltitle = {四川工人日报},
	author = {李眉},
	date = {1986-08-22},
}

@newspaper{宣言书1912,
	title = {女子参政同盟会宣言书},
	pages = {4},
	journaltitle = {时报},
	date = {1912-04-10},
}



@newspaper{章程1910,
	title = {四川会议厅暂行章程},
	pages = {“新章”,第 1—2 页},
	journaltitle = {广益丛报},
	date = {1910-09-03},
	Issue={第8年第19期},
}


@unpublished{方明东2000,
	title = {罗隆基政治思想研究(1913—1949)},
	institution = {北京师范大学历史系},
	type = {博士学位论文},
	author = {方明东},
	date = {2000},
	pages={67}
}


@unpublished{任东来2000,
	location = {天津},
	title = {对国际体制和国际制度的理解和翻译},
	type = {“全球化与亚太区域化国际研讨会”论文},
	pages = {9},
	author = {任东来},
	date = {2000-06},
}

@unpublished{周长山2004,
	title = {对汉代政治的若干考察——以郡(国)县体制为中心},
	shorttitle = {对汉代政治的若干考察},
	pages = {97--98},
	institution = {北京大学},
	type = {博士后研究报告},
	author = {周长山},
	date = {2004},
}

@unpublished{傅良佐1917,
	title = {傅良佐致国务院电},
	date = {1917-09-15},
	pages={李劼人档案,无编号,中共四川省委**部档案室藏}
}

@unpublished{党外记录1950,
	title = {党外人士座谈会记录},
	date = {1950-07},
	pages={北洋档案 1011-5961,中国第二历史档案馆藏}
}



@online{王明亮1998,
	title = {关于中国学术期刊标准化数据库系统工程的进展},
	url = {http://www.cajcd.cn/pub/wml.txt/980810-2.html},
	author = {王明亮},
	urldate = {1998-10-04},
	date = {1998-08-16},
}


@article{扬之水2006,
	title = {两宋茶诗与茶事},
	url = {http://www.literature.org.cn/Article.asp?ID=199},
	number = {1},
	journaltitle = {文学遗产通讯},
	author = {扬之水},
	urldate = {2007-09-13},
	date = {2006},
	issue = {网路版试刊},
}



@book{乔启明1934,
	location = {南京},
	title = {江宁县淳化镇乡村社会之研究},
	publisher = {金陵大学农学院},
	author = {乔启明},
	date = {1934},
}

@incollection{马俊亚2004,
	location = {北京},
	title = {民国时期江宁的乡村治理},
	pages = {352},
	booktitle = {中国农村治理的历史与现状:以定县、邹平和江宁为例},
	publisher = {社会科学文献出版社},
	author = {马俊亚},
	editor = {徐秀丽},
	date = {2004},
}


@book{汤志钧1979,
	location = {北京},
	title = {章太炎年谱长编},
	volume = {下册},
	publisher = {中华书局},
	author = {汤志钧},
	date = {1979},
}

@unpublished{章太炎1925,
	title = {在长沙晨光学校演说},
	author = {章太炎},
	date = {1925-10},
}

@book{邱陵1984,
	location = {哈尔滨},
	title = {书籍装帧艺术简史},
	publisher = {黑龙江人民出版社},
	editor = {邱陵},
	editortype={编著},
	date = {1984},
	pages={28-29}
}

@book{张秀民1989,
	location = {上海},
	title = {中国印刷史},
	publisher = {上海人民出版社},
	author = {张秀民},
	date = {1989},
	pages={531-532}
}

@book{汪荣祖1992,
	location = {南昌},
	title = {陈寅恪评传},
	publisher = {百花洲文艺出版社},
	author = {汪荣祖},
	date = {1992},
	pages={262-265}
}

@book{金毓黻1992,
	entrysubtype={classic},
	location = {沈阳},
	title = {静晤室日记},
	volume={第1册,1920年3月18日},
	publisher = {辽沈书社},
	author = {金毓黻},
	date = {1993},
	pages={11}
}

@book{李鹏程1992,
	location = {北京},
	title = {当代文化哲学沉思},
	volume={第1册,1920年3月18日},
	publisher = {人民出版社},
	author = {李鹏程},
	date = {1994},
	pages={“序言”,第1页}
}

@book{安徽会馆2001,
	location = {北京},
	title = {北京安徽会馆志稿},
	publisher = {北京燕山出版社},
	author = {{北京市宣武区档案馆编、王灿炽纂}},
	date = {2001},
}

@book{Starn1992,
	author = {Randolph Starn and Loren Partridge},
	title = {The Arts of Power: Three Halls of State in Italy, 1300-1600},
    address = {Berkeley},
	publisher = {California University Press},
	year = {1992},
    pages={19-28}
}


@book{Polo1997,
	address = {Hertfordshire},
	title = {The travels of {Marco} {Polo}},
	language = {en},
	publisher = {Cumberland House},
	author = {Polo, Marco},
	translator = {{William Marsden}},
	year = {1997},
    pages={58,88}
}

@collection{Aston1987,
	title = {The Brenner Debate},
	isbn = {978-0-521-34933-8},
	shorttitle = {The Brenner Debate},
	pages = {356},
	publisher = {Cambridge University Press},
	editor = {Aston, Trevor Henry and Philpin, C. H. E.},
	date = {1987-03-30},
	note = {Google-Books-{ID}: u23ilntsjfMC},
}

@incollection{Schfield1983,
	address = {Cambridge, Mass.},
	title = {The impact of scarcity and plenty on population change in {England}},
	booktitle = {Hunger and history: {The} impact of changing food production and consumption pattern on society},
	publisher = {Cambridge University Press},
	author = {Schfield, R. S.},
	editor = {Rotberg, R. I. and Rabb, T. K.},
    editortype={editor},
	year = {1983},
	pages = {55--88},
}

@article{Chamberlain1993,
	title = {On the search for civil society in {China}},
	volume = {19},
	doi = {10/d5gx4j},
	number = {2},
	journal = {Modern China},
	author = {{Heath B. Chamberlain}},
	date = {1993-04},
	pages = {199--215},
}

@article{thu.app.i.2.5:05,
	title = {On the Search for Civil Society in China},
	volume = {19},
	issn = {0097-7004},
	url = {https://doi.org/10.1177/009770049301900206},
	doi = {10.1177/009770049301900206},
	pages = {199--215},
	number = {2},
	journaltitle = {Modern China},
	shortjournal = {Mod. China},
	author = {Chamberlain, Heath B.},
	urldate = {2023-03-28},
	date = {1993-04},
	langid = {american},
	note = {Publisher: {SAGE} Publications Inc},
}


@incollection{thu.app.i.2.5:04,
	location = {Cambridge},
	title = {The Impact of Scarcity and Plenty on Population Change in England, 1541-1871},
	url = {https://www.jstor.org/stable/203704},
	pages = {265--291},
	booktitle = {Hunger and History: The Impact of Changing Food Production and Consumption Patterns on Society},
	publisher = {Cambridge University Press},
	author = {Schofield, Roger},
	editor = {Rotberg, Robert I. and Rabb, Theodore K.},
	urldate = {2023-03-28},
	date = {1983},
	langid = {american},
	note = {Publisher: The {MIT} Press},
}

@book{thu.app.i.2.5:02,
	location = {Hertfordshire},
	title = {The travels of Marco Polo},
	publisher = {Cumberland House},
	author = {Polo, Marco},
	translator = {Marsden, William},
	date = {1997},
	langid = {american},
}

@book{thu.app.i.2.5:01,
	title = {Arts of Power: Three Halls of State in Italy, 1300-1600},
	isbn = {978-0-520-07383-8},
	shorttitle = {Arts of Power},
	pages = {404},
	publisher = {University of California Press},
	author = {Starn, Randolph and Partridge, Loren W.},
	date = {1992-01-01},
	langid = {american},
}

\end{filecontents}

\addbibresource{\jobname.bib}


\begin{document}

\title{综合性期刊文献引证技术规范}
\maketitle

\section{注释体例}

\subsection{引文的标示}

无论是直接引文还是间接引文,正文中的注释号统一置于包含引文的句子或段落标点符号之后(对专门词语做注释时除外,注释号可紧随其后)。
每条注释不分段。所有注释均按照前后顺序逐条依次编排。注释可以放置于当页下(脚注),也可放置于文末(尾注),各刊任选其一。
注释序号用①,②,③……,或用右上角码1,2,3……,不管选择哪一种,注释前的编号应与正文中引文后的序号一致。通过注释序号将正文中引文与页下和文末注
释准确对应联系,完成文献引证的功能。

\subsection{注释的标注}

资料性注释中各项目的标注,应以被引证文献的版权页为准。

具体标注情况如下。

\subsubsection{专著}
1.专著:责任者(责任方式)/题名/卷册/出版地/出版者/出版年/页码。责任方式与题名之间一律用冒号;责任方式为“著”可省略。

示例1:

戴裔煊:《〈明史·佛郎机传〉笺正》,北京:中国社会科学出版社,1984年,第6页。
\footfullcite{戴裔煊1984} \qd{注意:这是正常情况}

示例2:

许毅等著:《清代外债史论》,北京:中国财政经济出版社,1996年,第95页。
\footfullcite{许毅1996} \qd{注意:这是正常情况}

示例:

马士:《东印度公司对华贸易编年史》,区宗华译,广州:中山大学出版社,1991年,第4、5卷合订本,第429-431页。
\footfullcite{马士1991}\qd{注意:“第4、5卷合订本”这种年份后的特殊信息,直接写在year里面,不用date。写成year=\{1991年,第4、5卷合订本\}}

\subsubsection{析出文献}
2.析出文献:析出文献责任者/析出文献题名/(见)文集责任者(责任方式)/文集题名/卷册/出版地/出版者/出版年/页码。文集责任者前的“见”在不产
生歧异的情况下,可省略。

示例1:

汪子春:《中国养蚕科学技术的发展和传播》,见自然科学研究所编:《中国古代科技成就》,北京:中国青年出版社,1978年,第382-391页。
\footfullcite{汪子春1978},\qd{注意:编者用bookauthor或editor输入,若用bookauthor,则编者类型无法设置,若用editor,设置或不设置editortype均可,不设置会自动输出“编”}。

示例2:

鲁迅:《中国小说的历史的变迁》,《鲁迅全集》第9册,北京:人民文学出版社,1981年,第325页。
\footfullcite{鲁迅2005} \qd{注意:“第9册”卷册信息,若是卷可以则设置volume=\{9\},若是册则可以则设置number=\{9\},当然也可以在volume或number中直接给出“第9册”这种原始信息}

文集责任者与析出文献责任者为同一作者时,可用省去责任者。

示例1:

唐振常:《师承与变法》,《识史集》,上海:上海古籍出版社,1997年,第65页。
\footfullcite{唐振常1997}\qd{注意:这是正常情况,文集作者相同或不存在则不输出}

示例2:

李志刚:《早期传教士在港创办的第一份中文报刊――遐迩贯珍>》,《***与近代中国文化》,台北:宇宙光出版社,1989年,第135-143页。
\footfullcite{李志刚1989}

书信、档案资料等应标识析出文献的形成时间。

示例1:

蔡元培:《复孙毓修函》,1911年6月3日,见高平叔、王世儒编注:《蔡元培书信集》上册,杭州:浙江教育出版社,2000年,第99页。
\footfullcite{蔡元培2000} \qd{注意:形成日期用eventdate或origdate表示,编者用editor输入,编注可以直接在editortype输入}

著作、文集的序言、前言有单独的标题,可作为析出文献来标注。

示例1:
黄仁宇:《为什么称为“中国大历史”?——中文版自序》,《中国大历史》,北京:三联书店,1997年,第2页。
\footfullcite{黄仁宇1997}

示例2:

楼适夷:《读家书,想傅雷(代序)》,见傅敏编:《傅雷家书》(增补本),北京:三联书店,1988年,第2页。
\footfullcite{楼适夷1988}  \qd{注意:“(增补本)”这种与书名间没有逗号的信息放volume里面}


\subsubsection{古籍}
3.古籍:

古籍既有传统的刻本、抄本,也有具有现代出版形式的标点本、整理本、影印本,情况比较复杂,可根据古籍形式的不同选择标注方式。

(1)抄本或刻本

标注顺序:责任者与责任方式/文献题名(卷次、篇名、部类名)/版本/页码。其中篇名、部类名为选项。原刻本标注版本信息,页码有两面,标注时应注明,用
a、b或上、下区分。
示例:

姚际恒:《古今伪书考》卷三,光绪三年苏州文学山房活字本,第九页a。
\footfullcite{姚际恒光绪三}\qd{注意:“卷三”这种与“第3卷”这种标准信息不一致的,直接将其原样放在volume里面,比如volume=\{卷三\};“光绪三年苏州文学山房活字本”这种特殊信息可以放到edition里面也可以放到year里面}

(2)点校本、整理本、影印本
点校本、整理本、影印本古籍为现代出版形式,引用时可参照现代著作(包括析出文献)的标注方式,其标注顺序:责任者与责任方式/文献题名(卷次、篇名、部
类)(选项)/出版地点/出版者/出版时间/卷册、页码。

示例1:
《清史稿》卷四八六《文苑三·吴汝纶传》,北京:中华书局,1977年标点本,第44册,第13444页。
\footfullcite{清史稿1977}\qd{注意:版本一般放在书名后面,这种特殊的情况,直接放到year中,比如:year=\{1977年标点本\},或者放到edition里面,但要设置entrysubtype=\{classic\},此时年份和版本分别设置为:date=\{1977\}和edition=\{标点本\}}

示例2:
屈大均:《广东新语》卷九《东莞城隍》,北京:中华书局,1985年标点本,第215页。
\footfullcite{屈大均1985}

影印本古籍通常采用缩印的方式,为便于读者查找,也可标明上、中、下栏(选项)。

示例:
杨钟羲:《雪桥诗话续集》卷五,沈阳:辽沈书社,1991年影印本,上册,第461页下栏。
\footfullcite{杨钟羲1991}

常用基本典籍,官修大型典籍以及书名中含有作者姓名的文集可不标注作者,如《论语》、二十四史、《资治通鉴》《全唐文》《册府元龟》《明实录》《四库全书总
目提要》《陶渊明集》等。

示例2:
《太平御览》卷六九《服章部七》引《魏台访议》,北京:中华书局,1985年影印本,第3册,第3080页下栏。
\footfullcite{太平御览1985}

编年体典籍,如需要,可注出文字所属之年月甲子(日)。
示例:

《资治通鉴》卷二,唐高宗永徽六年十月乙卯,北京:中华书局,1956年标点本,第?册,第6293页。
\footfullcite{资治通鉴}

唐宋时期的地方志多系私人著作,可标注作者;明清以后的地方志一般不标注作者,书名其前冠以修纂成书时的年代(年号);民国地方志,在书名前冠加“民国”二
字。新影印(缩印)的地方志可采用新页码。

示例1:
乾隆《嘉定县志》卷十二《风俗》,第七页a。
\footfullcite{嘉定县志} \qd{注意:成书年代,放到origyear中,比如:origyear=\{乾隆\},同时条目的entrysubtype要设置为classic}

示例2:
万历《广东通志》卷十五《郡县志二·广州府·城池》,《稀见中国地方志汇刊》第42册,北京:中国书店,1992年,第367页。
\footfullcite{广东通志}



(3)古籍中的析出文献

析出文献:析出文献责任者/析出文献题名/文集责任者与责任方式/文集题名/卷次/出版信息与版本/页码。

示例1
管志道:《答屠仪部赤水丈书》,《续问辨牍》卷二,《四库全书存目丛书·子部》第88册,济南:齐鲁书社,1997年,第73页。
\footfullcite{管志道1997}

示例2:
何承矩:《上太宗论塘泊屯田之利》,载赵汝愚编:《宋名臣奏议》卷一〇五,《文渊阁四库丛书》第432册,上海:上海古籍出版社,1987年,第284页。
\footfullcite{何承矩1987}


(4)引用古籍与夹注的使用

引用先秦诸子等常用经典古籍,可使用夹注,夹注应使用不同于正文的字体。
 示例 1:

 庄子说惠子非常博学,“惠施多方,其书五车。”\citejz{庄子天下}

 示例 2:

 天神所具有道德,也就是“保民”、“裕民”的道德;天神所具有的道德意志,代表的是人民的意志。这也就是所谓“天聪明自我民聪明,天明畏自我民明畏”\citejz{尚书皋陶谟},“民之所 欲,天必从之”\citejz{尚书泰誓}。


\subsubsection{连续出版物(期刊、报纸等)}
4.连续出版物(期刊、报纸等)

(1)期刊:责任者/文章题名/期刊题名/出版年、卷期或出版日期。引用中国大陆以外出版的中文期刊时,应在期刊题名后括注出版地。如果需要,引用期刊时,
也可标注页码。

示例1:
吴艳红:《明代流刑考》,《历史研究》2000年第6期。
\footfullcite{吴艳红2000} \qd{注意:正常的期刊把卷期设置成数字即可}

示例2:
吴虞:《家族制度为专制主义之根据论》,《新青年》第2卷第6号,1917年2月1日。
\footfullcite{吴虞1917} \qd{注意:特殊的期刊卷期在年份前的,将将这些信息放到issue中}

中国大陆以外出版的中文期刊应在刊名后括注出版地点;刊名与其他期刊相同,也可括注出版地点,附于刊名后,以示区别。

示例1:
王晴佳:《中国二十世纪史学与西方--论现代历史意识的产生》,《新史学》(台北)第9卷第1期,1998年3月,第55-82页。
\footfullcite{王晴佳1998}\qd{注意:地点直接放在location中,若跟卷期一起,也可以不要location直接原样放到issue中}

示例2:
魏丽英:《论近代西北人口波动的主要原因》,《社会科学》(兰州)1990年第6期。
\footfullcite{魏丽英1990}

示例3:
费成康:《葡萄牙人如何进入澳门问题辨证》,《社会科学》(上海)1999年第9期。
\footfullcite{费成康1999}

示例4:
董一沙:《回忆**董希文》,《传记文学》(北京)2001年第3期。
\footfullcite{董一沙2001}

示例5:
李济:《创办史语所与支持安阳考古工作的贡献》,《传记文学》(台北)第28卷第1期,1976年1月。
\footfullcite{李济1976}


同一种期刊有两个以上的版别时,引用时须注明版别。

示例1:
黄义豪:《评黄龟年四劾秦桧》,《福建论坛》(文史哲版)1997年第3期。
\footfullcite{黄义豪1997}\qd{注意:版别放在series中,比如:series = \{文史哲版\}}

示例2:
苏振芳:《新加坡推行儒家**道德教育的社会学思考》,《福建论坛》(经济社会版)1996年第3期。
\footfullcite{苏振芳1996}

示例3:
叶明勇:《英国议会圈地及其影响》,《武汉大学学报》(人文科学版)2001年第2期。
\footfullcite{叶明勇2001}

示例4:
倪素香:《德育学科的比较研究与理论探索》,《武汉大学学报》(社会科学版)2002年第4期
\footfullcite{倪素香2002}

(2)报纸:责任者/文章题名/报纸题名/出版日期或卷期(附出版年月)/版次。

示例1:
李眉:《李劼人轶事》,《四川工人日报》1986年8月22日,第2版。
\footfullcite{李眉1986}\qd{注意:报纸类型entrytype最好用newspaper}

示例2:
《女子参政同盟会宣言书》,《时报》1912年4月10日,第4版。
\footfullcite{宣言书1912}

早期报纸无版次,可标识栏目及页码(选注项)。

示例:
《四川会议厅暂行章程》,《广益丛报》第8年第19期,1910年9月3日,“新章”,第1—2页。
\footfullcite{章程1910}\qd{注意:复杂的页面信息直接放页码里面}



\subsubsection{未刊文献}
6.未刊文献:责任者/未刊文献题名/文献属性/编号/收藏单位/页码(选项)。

示例1:
方明东:《罗隆基政治思想研究(1913-1949)》,博士学位论文,北京师范大学历史系,2000年,第67页。
\footfullcite{方明东2000}\qd{注意:未刊文献用unpublished,论文类型放type中}

示例2:
任东来:《对国际体制和国际制度的理解和翻译》,“全球化与亚太区域化国际研讨会”论文,天津,2000年6月,第9页。
\footfullcite{任东来2000}

示例3:
周长山:《对汉代政治的若干考察----以郡(国)县体制为中心》,博士后研究报告,北京大学,2004年,第97-98页。
\footfullcite{周长山2004}\qd{注意:年份后面的复杂信息可以直接放pages里面}

示例4:
《党外人士座谈会记录》,1950年7月,李劼人档案,无编号,中共四川省委**部档案室藏。
\footfullcite{党外记录1950}

示例5
《傅良佐致国务院电》,1917年9月15日,北洋档案1011-5961,中国第二历史档案馆藏。
\footfullcite{傅良佐1917}


\subsubsection{电子文献}

7.电子文献

电子文献包括以数码方式记录的所有文献(含以胶片、磁带等介质记录的电影、录影、录音等音像文献)。其标注项目与顺序:责任者/电子文献题名/更新或修改日
期/获取和访问路径/引用日期(任选)。

示例1:
王明亮:《关于中国学术期刊标准化数据库系统工程的进展》,1998年8月16日,\url{http://www.cajcd.cn/pub/wml.txt/980810-2.html}, 1998年10月4日。
\footfullcite{王明亮1998}\qd{注意:一般电子文献直接用online}

示例2:
扬之水:《两宋茶诗与茶事》,《文学遗产通讯》(网路版试刊)2006年第1期,\url{http://www.literature.org.cn /Article.asp?ID=199},2007年9月13日。
\footfullcite{扬之水2006}\qd{注意:aritcle电子文献直接用aritcle}



\subsubsection{转引文献}
8.转引文献
责任者/原文献题名/原文献版本信息/原页码(或卷期)/转引文献责任者/转引文献题名/版本信息/页码。

示例1:
乔启明:《江宁县淳化镇乡村社会之研究》,南京:金陵大学农学院,1934年,第17页。转引自马俊亚:《民国时期江宁的乡村治理》,见徐秀丽主编:《中国农村治
理的历史与现状:以定县、邹平和江宁为例》,北京:社会科学文献出版社,2004年,第352页。
\footnote{\fullinnercite{乔启明1934},转引自\fullcite{马俊亚2004}}
\qd{注意:转引可以用两个条目实现,比如:}
\verb|\footnote{\fullinnercite{乔启明1934},转引自\fullcite{马俊亚2004}}|

示例2:
章太炎:《在长沙晨光学校演说》,1925年10月,转引自汤志钧:《章太炎年谱长编》下册,北京:中华书局,1979年,第823页。
\footnote{\fullinnercite{章太炎1925},转引自\fullcite{汤志钧1979}}


\subsubsection{间接引文}
10.间接引文

间接引文通常以“参见”“参阅”或“详见”等引领词引导标注项目,应标识出具体参考引证的起止页码或章节。“参见”“参阅”或“详见”后不加冒号。标注项
目、顺序与格式同直接引文。

示例:
参见邱陵编著:《书籍装帧艺术简史》,哈尔滨:黑龙江人民出版社,1984年,第28-29页;张秀民:《中国印刷史》,上海:上海人民出版社,1989年,第531-32页。
\footnote{参见\fullcite{邱陵1984,张秀民1989}}


\subsubsection{内容性注释中的引文}
11.内容性注释中的引文

内容性注释是指对正文中的术语、概念、观点和资料进行进一步的解释、辨析或评论的文字。很多情况下,内容性注释中又出现引文,无论是直接引文,还是间接引
文,都需要标明资料出处,标注项目与顺序与资料性注释相同

示例1:
汪荣祖已注意到陈寅恪对胡适推崇《马氏文通》和用西洋哲学条理中国古代思想的批评。参见汪荣祖:《陈寅恪评传》,南昌:百花洲文艺出版社,1992年,第262-265页。
\footnote{汪荣祖已注意到陈寅恪对胡适推崇《马氏文通》和用西洋哲学条理中国古代思想的批评。参见\fullcite{汪荣祖1992}}

示例2:
金毓黻曾说,“论学首贵析理”,而论事则“需兼及情与势;情浃而势合,施之于事,无不允当也”。见金毓黻:《静晤室日记》第1册,1920年3月18日,沈阳:辽沈书社,1993年,第11页。
\footnote{金毓黻曾说,“论学首贵析理”,而论事则“需兼及情与势;情浃而势合,施之于事,无不允当也”。见\fullcite{金毓黻1992}}


\subsubsection{引证著作、文集的序言、引论、前言}
12.引证著作、文集的序言、引论、前言。

(1)序言作者与著作、文集责任者相同:

示例:
李鹏程:《当代文化哲学沉思》,北京:人民出版社,1994年,“序言”,第1页。
\footfullcite{李鹏程1992}


(2)责任者层次关系复杂时,也可以采用内容性注释的方式,通过叙述表明对序言的引证。为了表述紧凑,可将出版信息括注起来。

示例:
见戴逸为北京市宣武区档案馆编、王灿炽纂:《北京安徽会馆志稿》(北京:北京燕山出版社,2001年)所作的序,第2页。
\footnote{见戴逸为北京市宣武区档案馆编、王灿炽纂:\citetitle{安徽会馆2001},(\citepub{安徽会馆2001})所作的序,第2页。}

\subsubsection{外文文献}
5.外文文献

引证外文文献,原则上以该文种通行的引证标注方式为准。
引证英文文献的标注项目与顺序与中文相同。责任者与题名间用英文逗号,著作题名为斜体,析出文献题名为正体加英文引号,出版日期为全数字标注,责任方式、卷
册、页码等用英文缩略方式。

示例1:
Randolph Starn and Loren Partridge,\textit{The Arts of Power:Three Halls of State in Italy, 1300-1600}, Berkeley: California University Press, 1992, pp. 19-28.
\footfullcite{Starn1992}

示例2:
M. Polo,\textit{The Tr**els of Marco Polo},trans.by William Marsden, Hertfordshire: Cumberland House, 1997, pp. 55, 88.
\footfullcite{Polo1997}

示例3:
T. H. Aston and C. H. E. Phlipin (eds.),\textit{The Brenner Debate}. Cambridge: Cambridge University Press, 1985, p. 35.
\footfullcite{Aston1987}

示例4:
R. S. Schfield, “The Impact of Scarcity and Plenty on Population Change in England,” in R. I.Rotberg and T. K. Rabb (eds.), \textit{Hunger and History: The Impact of Changing Food
Production and Consumption Pattern on Societ}, Cambridge: Cambridge University Press, 1983, p. 79.
\footfullcite{Schfield1983}

示例5:
Heath B. Chamberlain, “On the Search for Civil Society in China”, \textit{Modern China}, vol. 19, no. 2 (April 1993), pp. 199-215.
\footfullcite{Chamberlain1993}


\newpage
\printbibliography[heading=bibliography]
\end{document} 