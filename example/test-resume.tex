

% !Mode:: "TeX:UTF-8"
% 用于测试ucasthesis中应用biblatex-gb7714-2015的情况
%

\documentclass[twoside]{article}
    \usepackage{ctex}
    \usepackage{xcolor}
    \usepackage[colorlinks,citecolor=blue]{hyperref}
    \usepackage[style=gb7714-2015]{biblatex}
    \usepackage{geometry}

    \begin{filecontents}{resume.bib}
    @article{ref-1-1-Yang,
    author    = {Yang, Y and Ren, T L and Zhang, L T and others},
    title     = {Miniature microphone with silicon-based ferroelectric thin films},
    journal   = {Integrated Ferroelectrics},
    date      = {2003},
    pages      = {229-235},
    volume      = {52},
    annotation      = {SCI 收录, 检索号:758FZ},
    AUTHOR+an={1=thesisauthor},
    OPTIONS = {maxbibnames=1,minbibnames=1}
    }

    @article{ref-2-1-杨轶,
    author    = {杨轶 and 张宁欣 and 任天令 and others},
    title     = {硅基铁电微声学器件中薄膜残余应力的研究},
    journal   = {中国机械工程},
    pages= {1289-1291},
    date      = {2005},
    volume      = {16},
    number ={14},
    annotation ={EI 收录, 检索号:0534931 2907},
    AUTHOR+an={1=thesisauthor}
    }

    @patent{ref-8-1-任天令,
    author    = {任天令 and 杨轶 and 朱一平 and others},
    title     = {硅基铁电微声学传感器畴极化区域控制和电极连接的方法},
    number ={中国, CN1602118A},
    annotation      = {中国专利公开号.},
    }

    @patent{ref-9-1-Ren,
    author    = {Ren, T L and Yang, Y and Zhu, Y P and others},
    title     = {Piezoelectric micro acoustic sensor based on ferroelectric materials},
    number ={USA, No.11/215, 102},
    annotation      = {美国发明专利申请号.},
    }
    \end{filecontents}
    \addbibresource{resume.bib}
    %

    \begin{document}

	\begin{refsection}[resume.bib]
	\nocite{ref-8-1-任天令,ref-9-1-Ren}%
	\printbibliography[heading=subbibliography,title={研究成果}]
    \end{refsection}


	\begin{refsection}[resume.bib]
	\settoggle{bbx:gbtype}{false}%局部设置不输出文献类型和载体标识符
	\settoggle{bbx:gbannote}{true}%局部设置输出注释信息
	\setcounter{gbnamefmtcase}{1}%局部设置作者的格式为familyahead格式
    \makeatletter
    \renewcommand*{\mkbibnamegiven}[1]{%通过作者注释局部调整作者的格式需与bib配合
    \ifitemannotation{thesisauthor}
    {\ifbibliography{\textcolor{blue}{\textbf{#1}}}{#1}}%
    {#1}\ifbibliography{\ifitemannotation{corresponding}{\textsuperscript{*}}{}}{}%
    }
    \renewcommand*{\mkbibnamefamily}[1]{%
    \ifitemannotation{thesisauthor}
    {\ifbibliography{\textcolor{blue}{\textbf{#1}}}{#1}}
    {#1}}
    \def\blx@maxbibnames{2}
    \def\blx@minbibnames{2}
    %\defcounter{gbbiblocalcase}{1} %局部强迫中文本地化字符串
    %\defcounter{gbbiblocalcase}{2} %局部强迫英文本地化字符串
    \setlocalbibstring{andotherscn}{et al.}
    \setlocalbibstring{andothers}{等}
    \makeatother

	\nocite{ref-1-1-Yang,ref-2-1-杨轶}
	
	\setlength{\biblabelsep}{12pt}
	\printbibliography[heading=subbibliography,title={发表的学术论文}] %发表和录用的
	\end{refsection}

\end{document} 