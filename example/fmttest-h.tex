

    \section{测试: 中英文判断信息中存在编组时的情况}
\begin{refsection}
当责任者等需要判断中英文的信息中存在编组时的处理
\cite{IFLAI1977b--,IFLAI1977--}
\cite{r27-BenHadjAlaya-FekiA.2008-1-5,中国企业投资协会2014--,中国企业投资协会2015--}

\printbibliography[heading=subbibliography,title=【中英文判断信息中存在编组的测试】]
\end{refsection}


    \section{测试: 参考文献信息中存在\&等特殊字符的情况}
\begin{refsection}
文献中\cite{ref-replace-char}的booktitle域中含有\%,\&,\#符号,样式文件自动处理使其符合tex代码规则。
\printbibliography[heading=subbibliography,title=【处理参考文献信息中\&等特殊字符】]
\end{refsection}

\section{测试: 作者年制article中卷信息缺省时的标点情况}
\begin{refsection}
文献\cite{刘彻东1998-38-39}\cite{亚洲地质图编目组1978-194-208}
\cite{高光明1998-60-65}

\printbibliography[heading=subbibliography,title=【author-year style: article without volume】]
\end{refsection}

\section{测试: 标题中存在\textbackslash LaTeX\{\}等命令时的情况}

\begin{refsection}
文献\cite{Peebles2001-100-100}\cite{赵凯华1995--}\cite{蒋有绪1998--}

\printbibliography[heading=subbibliography,title=【title with \textbackslash LaTeX\{\}】]
\end{refsection}





