\usepackage{expl3,etoolbox,ifthen,xstring}
\usepackage{xltxtra,mflogo,texnames}
\usepackage[zihao=5]{ctex}
\ctexset{today=old}
\let\kaiti=\kaishu
\usepackage{xeCJKfntef}

\usepackage{xcolor}
\colorlet{examplefill}{yellow!80!black}
\definecolor{graphicbackground}{rgb}{0.96,0.96,0.8}
\definecolor{codebackground}{rgb}{0.9,0.9,1}
\definecolor{gbsteelblue}{RGB}{70,130,180}
\definecolor{gborange}{RGB}{255,138,88}
\definecolor{gbblue}{RGB}{23,74,117}
\definecolor{gbforestgreen}{RGB}{21,122,81}
\definecolor{gbyellow}{RGB}{255,185,88}
\definecolor{gbgrey}{RGB}{200,200,200}
\colorlet{gblabelcolor}{violet}
\colorlet{gbemphcolor}{blue!60!black}

%定义版面,showframe,
\usepackage[paperwidth=210mm,paperheight=290mm,left=25mm,right=25mm,top=25mm, bottom=20mm,showcrop]{geometry}%,showframe
\renewcommand{\baselinestretch}{1.35}
%页面布局的标尺
\usepackage[type=none]{fgruler}
%[unit=cm,type=lowerleft,showframe=true,hshift=3cm,vshift=2cm]
\rulerparams{}{}{gray!50}{}{0.4pt}
\fgrulerdefnum{}\fgrulercaptioncm{}%fgruler加数字后,导致基线对齐出现问题,所以这里去掉

\newlength{\skipheadrule}
\deflength{\skipheadrule}{3.5pt}
\newlength{\skipfootrule}
\deflength{\skipfootrule}{5.5pt}
\newlength{\ruletotalen}
\deflength{\ruletotalen}{\textheight}
\newlength{\ruleraised}
\deflength{\ruleraised}{\headsep+\textheight}
\usepackage{fancyhdr}
\fancyhf{}
\fancyhead[LO]{%
\raisebox{-\skipheadrule}{%
\raisebox{-\headsep}[0pt][0pt]{\makebox[0pt][l]{\ruler{rightup}{\linewidth}}}%
\raisebox{-\ruleraised}[0pt][0pt]{\makebox[0pt][r]{\ruler{upleft}{\ruletotalen}}}%
}%HEAD LEFT%
}
\fancyhead[LE]{%
\raisebox{-\skipheadrule}{%
\raisebox{-\headsep}[0pt][0pt]{\makebox[0pt][l]{\ruler{rightup}{\linewidth}}}%
\raisebox{-\ruleraised}[0pt][0pt]{\makebox[0pt][r]{\ruler{upleft}{\ruletotalen}}}%
}\leftmark%HEAD LEFT%
}
\fancyhead[RO]{%
%HEAD RIGHT%
\raisebox{-\skipheadrule}{%
\hfill\makebox[0pt][l]{\raisebox{-\ruleraised}[0pt][0pt]{\ruler{downright}{\ruletotalen}}\hss}%
}}
\fancyhead[RE]{%
\rightmark%HEAD RIGHT%
\raisebox{-\skipheadrule}{%
\hfill\makebox[0pt][l]{\raisebox{-\ruleraised}[0pt][0pt]{\ruler{downright}{\ruletotalen}}\hss}%
}}
\fancyhead[CO]{%
符合GB/T 7714-2015标准的biblatex参考文献样式%HEAD CENTER
}
\fancyfoot[L]{%
\raisebox{-\skipfootrule}{%
\raisebox{\footskip}[0pt][0pt]{\makebox[0pt][l]{\ruler{rightdown}{\linewidth}}}
}%FOOT LEFT
}
\fancyfoot[C]{%
\thepage%FOOT CENTER
}
\fancyfoot[R]{%
%FOOT RIGHT
}
\renewcommand{\headrulewidth}{0.4pt}
\renewcommand{\footrulewidth}{0pt}
\pagestyle{fancy}


%超链接书签功能,选项去掉链接红色方框
\usepackage[colorlinks=true,%
pdfstartview=FitH,allcolors=gbemphcolor]{hyperref}
%linkcolor=gbblue,anchorcolor=gbblue,citecolor=gbblue
%linkcolor=black,linkcolor=green,blue,red,cyan, magenta,
%yellow, black, gray,white, darkgray, lightgray, brown,
%lime, olive, orange, red,purple, teal, violet.
%CJKbookmarks,bookmarksnumbered=true,
\usepackage{titleref} %标题引用

%标题格式设置
\usepackage{titlesec}
%\titlespacing*{hcommandi}{hlefti}{hbefore-sepi}{hafter-sepi}[hright-sepi]
\titlespacing*{\section}{0pt}{\baselineskip}{0.5\baselineskip}
\titlespacing*{\subsection}{0pt}{0.5\baselineskip}{0.5\baselineskip}
\titlespacing*{\subsubsection}{0pt}{0.5\baselineskip}{0pt}
\titlespacing{\paragraph}{2em}{0.5\baselineskip}{1em}

%参考文献
\usepackage[backend=biber,style=gb7714-2015,gbalign=center%,gbtype=true%
]{biblatex}%,backref=true%
\addbibresource[location=local]{example/example.bib}
\setlength{\bibitemsep}{1pt}
%\defbibheading{bibliography}[\bibname]{%
%%\phantomsection%解决链接指引出错的问题,相当于加入了一个引导点
%%\addcontentsline{toc}{subsection}{#1}
%	\centering\subsubsection*{#1}}%


%目录,图/表/例目录,图表题注
\usepackage{subfigure}
\usepackage[subfigure]{tocloft} %注意其与titletoc共用时分页会有问题
\usepackage{ccaption}
\captiondelim{. } %图序图题中间的间隔符号
\captionnamefont{\zihao{-5}\heiti} %图序样式
\captiontitlefont{\zihao{-5}\heiti} %图题样式
\captionwidth{0.8\linewidth} %标题宽度
\changecaptionwidth
\captionstyle{\centering} %\captionstyle{<style>} style are: \centering, \raggedleft or \raggedright
%\precaption{\rule{\linewidth}{0.4pt}\par}
%\postcaption{\vspace{-1cm}}
\setlength{\belowcaptionskip}{2pt}%设置caption上下间距
\setlength{\abovecaptionskip}{0pt}
%\setlength{\abovelegendskip}{0pt} %设置legend上下间距
%\setlength{\belowlegendskip}{0pt}
%新的浮动体设置,\centerline{}
\newcommand{\listegcodename}{\zihao{4}示~~例\thispagestyle{plain}}%listegcodename,新环境目录的标题
\newcommand{\egcodename}{例}%egcodename,新环境标题的图序
\newfloatlist{egcode}{loc}{\listegcodename}{\egcodename}%loc,写入条目的文件的扩展名
\newfixedcaption{\codecaption}{egcode}%egcode,环境名

%目录命令
\setlength{\cftbeforetoctitleskip}{\baselineskip}
\setlength{\cftaftertoctitleskip}{0.5\baselineskip}
\setlength{\cftbeforeloftitleskip}{\baselineskip}
\setlength{\cftafterloftitleskip}{0.5\baselineskip}
\setlength{\cftbeforelottitleskip}{\baselineskip}
\setlength{\cftafterlottitleskip}{0.5\baselineskip}
\setlength{\cftbeforeloctitleskip}{\baselineskip}
\setlength{\cftafterloctitleskip}{0.5\baselineskip}
%\renewcommand\contentsname{\hfill 目~~ 录 \hfill \hspace{1cm}} %用这一句也是一样的。
\renewcommand{\cfttoctitlefont}{\heiti}
\renewcommand{\cftaftertoctitle}{}
\renewcommand{\cftloftitlefont}{\heiti}
\renewcommand{\cftafterloftitle}{}
\renewcommand{\cftlottitlefont}{\heiti}
\renewcommand{\cftafterlottitle}{}
\renewcommand{\cftloctitlefont}{\heiti}
\renewcommand{\cftafterloctitle}{}
\renewcommand{\contentsname}{\zihao{4}目~~录}
\renewcommand{\listfigurename}{\zihao{4}图~~片}
\renewcommand{\listtablename}{\zihao{4}表~~格}
\renewcommand{\cftsecfont}{\zihao{5}\heiti} %条目样式
\renewcommand{\cftsubsecfont}{\zihao{-5}\songti} %条目样式\fangsong
\renewcommand{\cftsubsubsecfont}{\zihao{-5}\kaiti} %条目样式
%−−−−−−−−−−设置egcode条目样式−−−−−−−−−−−−−−−−−−−−−−
%\renewcommand{\cftegcodeleader}{\leaders\hbox to 1em{\hss.\hss}\hfill}
\setlength{\cftbeforeegcodeskip}{0.1ex} %条目前的间距
\setlength{\cftegcodeindent}{0em} %条目缩进
\setlength{\cftegcodenumwidth}{2.5em} %条目标签宽度
\renewcommand{\cftegcodefont}{\color{gbemphcolor}\zihao{-5}}%条目样式\fangsong
\renewcommand{\cftegcodepresnum}{例}
\renewcommand{\cftegcodeaftersnum}{ }
\renewcommand{\cftegcodeaftersnumb}{~}
%\cftsetindents{egcode}{0em}{3em}
%\renewcommand{\cftegcodepagefont}{\bfseries}
%−−−−−−−−−−设置figure条目样式−−−−−−−−−−−−−−−−−−−−−−
%\newcommand{\cftfigfill}{\renewcommand{\cftdot}{$\diamond$}\cftdotfill{\cftdotsep}}
\setlength{\cftbeforefigskip}{0.1ex} %条目前的间距
\setlength{\cftfigindent}{0em} %条目缩进
\setlength{\cftfignumwidth}{2.5em} %条目标签宽度
\renewcommand{\cftfigfont}{\color{gbemphcolor}\zihao{-5}} %条目样式\heiti
\renewcommand{\cftfigpresnum}{图} %条目数字前的内容
\renewcommand{\cftfigaftersnum}{ } %条目数字后的内容
\renewcommand{\cftfigaftersnumb}{~} %条目数字后的第二个内容
%\renewcommand{\cftfigdotsep}{\cftdotsep} %连接符之间的宽度
%\renewcommand{\cftfigleader}{\bfseries\cftfigfill} %连接符粘连团
%\renewcommand{\cftfigpagefont}{\color{red}\zihao{-5}$\diamond$\itshape} %页码的样式
%\renewcommand{\cftfigafterpnum}{\color{red}$\diamond$} %页码后内容
%−−−−−−−−−−设置table条目样式−−−−−−−−−−−−−−−−−−−−−−
%\newcommand{\cfttabfill}{\renewcommand{\cftdot}{$\infty$}\cftdotfill{\cftdotsep}}
\setlength{\cftbeforetabskip}{0.1ex} %条目前的间距
\setlength{\cfttabindent}{0em} %条目缩进
\setlength{\cfttabnumwidth}{2.5em} %条目标签宽度
\renewcommand{\cfttabfont}{\color{gbemphcolor}\zihao{-5}} %条目样式
\renewcommand{\cfttabpresnum}{表} %条目数字前的内容
\renewcommand{\cfttabaftersnum}{ } %条目数字后的内容
\renewcommand{\cfttabaftersnumb}{~} %条目数字后的第二个内容
%\renewcommand{\cfttabdotsep}{\cftdotsep} %连接符之间的宽度
%\renewcommand{\cfttableader}{\bfseries\cfttabfill} %连接符粘连团
%\renewcommand{\cfttabpagefont}{\color{red}\zihao{-5}$\infty$\itshape} %页码的样式
%\renewcommand{\cfttabafterpnum}{\color{red}$\infty$} %页码后内容

\usepackage{pdfpages}%直接插入pdf文件页
\graphicspath{{egfigure/}{example/}}

%代码环境设置
\usepackage{listings}
\usepackage{tikz,pgf}
\usetikzlibrary{calc}

\newenvironment{example}[3][代码]%
{\list{}{\begingroup\codecaption{#2}\label{#3}\endgroup
\setlength{\topsep}{0pt}
\setlength{\partopsep}{0pt}
\setlength{\itemsep}{0pt}
\setlength{\parsep}{0pt}
\setlength{\leftmargin}{0pt}%
\setlength{\itemindent}{0pt}%
%\renewcommand*{\makelabel}[1]{\hss\llap{\footnotesize\color{orange}\bfseries##1}}
}\item[\footnotesize\color{gblabelcolor}\bfseries#1]\relax}
{\endlist}

\lstnewenvironment{texlist}%
{\lstset{% general command to set parameter(s)
%name=#1,
%label=#2,
%caption=\lstname,
linewidth=\linewidth,
breaklines=true,
%showspaces=true,
extendedchars=false,
columns=fullflexible,%flexible,
aboveskip=2pt,
boxpos=t,
rulesep=0pt,
frame=tb,
framesep=0pt,
rulecolor=\color{gblabelcolor},
fontadjust=true,
language=[LaTeX]TeX,
backgroundcolor=\color{gbyellow!3},%\color{yellow}, %背景颜色
numbers=left,
numberstyle=\tiny\color{gblabelcolor},
basicstyle=\footnotesize\ttfamily, % print whole listing small
keywordstyle=\bfseries\color{gbemphcolor},%\underbar,
% underlined bold black keywords
identifierstyle=, % nothing happens
commentstyle=\color{green!40!gray}, % white comments
stringstyle=\ttfamily\color{purple!50}, % typewriter type for strings
showstringspaces=false}% no special string spaces
}
{}

%定理环境设置
\usepackage[listings,theorems,most]{tcolorbox}
\tcbuselibrary{breakable}
\newcounter{myprop}\def\themyprop{\arabic{myprop}}
%一个强调显示
\newcommand{\bibliofmt}[1]{\medskip\textcolor{gbforestgreen}{\heiti#1}}

%序号如果带章节的话可以改为比如:\thesection.\arabic{myprop}
\tcbmaketheorem{property}{方法}
{enhanced jigsaw,breakable,pad at break*=1mm,left=2em,boxsep=0pt,
 colback=black!5,colframe=gborange,coltitle=gborange,
 theorem style=plain,fonttitle=\bfseries,arc=0mm,
%separator sign={\ $\blacktriangleright$},breakable,
%theorem style=plain,fonttitle=\bfseries\upshape, fontupper=\slshape,boxrule=0mm,arc=0mm, %
%coltitle=black,colback=green!50!yellow!15!white,colframe=blue!50,%
%description delimiters={}{},
%terminator sign={\ }
}{myprop}{pp}
%最后一个必须参数是prefix用来做label比如这里是pp:加上给出的标签名

\newtcbtheorem[]{refentry}{条目类型}
{breakable,pad at break*=1mm,enhanced jigsaw,left=2em,boxsep=0pt,
 colback=yellow!10!white,boxrule=0pt,frame hidden,
 borderline west={1.5mm}{-2mm}{gbforestgreen},
separator sign={\ $\blacktriangleright$},terminator sign={\ },
theorem style=plain,fonttitle=\bfseries,coltitle=gbforestgreen
%fontupper=\normalsize,boxrule=0mm,arc=0mm,breakable,
%coltitle=green!35!black,colbacktitle=green!15!white,
%colback=green!50!yellow!15!white,terminator sign={\ }
}{rfeg}
%最后一个必须参数是prefix用来做label比如这里是rfeg:加上给出的标签名


%标题区命令设置
\newcommand{\titleformanual}[1]{\def\biaotiudf{#1}}
\newcommand{\authorformanual}[1]{\def\zuozheudf{#1}}
\newcommand{\dateformanual}[1]{\def\riqiudf{#1}}
%\ifthenelse{\equal{\youwuudf}{\temp}}{true}{false}
\def\temp{}
\newcommand{\titleandauthor}{
\begin{center}
{\renewcommand{\thefootnote}{\fnsymbol{footnote}}\setlength{\baselineskip}{30pt}\heiti{\zihao{-2}{\biaotiudf}}\par}
%注意这里\par要放在花括号内才有效
\vspace*{0.3cm}
{\renewcommand{\thefootnote}{\arabic{footnote}}\kaishu{\zihao{4}{\zuozheudf}}\par}
\vspace*{0.2cm}
{\songti{\zihao{-4}{\riqiudf}}\par}
\end{center}
}
%脚注的数字带圈
\renewcommand{\thefootnote}{\textcircled{\tiny\arabic{footnote}}}


%--------------列表环境---------------------------------------------
\usepackage[inline]{enumitem} %重设list环境
\setlist[enumerate]{label=\bfseries\textcolor{gbemphcolor}{(\arabic*)},topsep=2pt,partopsep=0pt,parsep=0pt,%
align=left,leftmargin=0em,itemsep=0.5em,labelwidth=0.1em,itemindent=2.6em,listparindent=2em}%label=$\triangleright$,itemindent=1em
\setlist[itemize]{topsep=2pt,partopsep=0pt,parsep=0pt,%
leftmargin=3em,itemindent=0em}
\setlist[description]{font=\bfseries\textcolor{gbemphcolor},align=right,topsep=5pt,partopsep=0pt,parsep=0pt,%
itemsep=0pt,leftmargin=0em,itemindent=0em}%注意,font或format中的最后一个命令自动提取标签为其参数

\usepackage{longtable}

%自定义下划红线和背景颜色
\usepackage{ulem}
\newcommand\yellowback{\bgroup\markoverwith
{\textcolor{yellow}{\rule[-0.5ex]{2pt}{2.5ex}}}\ULon}
\newcommand\reduline{\bgroup\markoverwith
{\textcolor{red}{\rule[-0.5ex]{2pt}{0.4pt}}}\ULon}

%一些字符串格式化命令
\newcommand*{\verbatimfont}{\ttfamily}
\newrobustcmd*{\cnt}[1]{\mbox{\verbatimfont#1}}
\newrobustcmd*{\bibfield}[1]{\mbox{\verbatimfont#1}}
\newrobustcmd*{\opt}[1]{\mbox{\verbatimfont#1}}
\newrobustcmd*{\prm}[1]{%
  \ifblank{#1}
    {}
    {\mbox{%
       \ensuremath\langle
       \normalfont\textit{#1}%
       \ensuremath\rangle}}}

\usepackage{amssymb}

\newcommand{\HandRight}{$\bigstar$}
\newcommand{\zhongdian}[1]{\textcolor{gbemphcolor}{\HandRight\small\heiti#1}}
\newcommand{\pz}[1]{%定义pz为旁注命令
\marginpar[\flushright\small\youyuan\color{gbemphcolor}\footnotesize #1]{\youyuan\color{gbemphcolor}\small #1}}
\newcommand{\PZ}[1]{%定义pz为旁注命令
\marginpar[\flushright\small\youyuan\color{gbemphcolor}\footnotesize  #1]{\small\youyuan\color{gbemphcolor}\small #1}}
\newcommand{\qd}[1]{%定义qd为强调命令
\begin{quote}
  \small\youyuan\color{gbemphcolor}#1%blue!50!black\fangsong
\end{quote}}
\newcommand{\QD}[1]{%定义qd为强调命令
\begin{quote}
  \small\youyuan\color{gbemphcolor}#1
\end{quote}}
\newcommand{\bc}[1]{%定义补充信息
{\small\youyuan\color{gbemphcolor}#1}} %orange,brown,purple,teal,gbblue,olive,cyan
\newcommand{\BC}[1]{%定义补充信息
{\small\youyuan\color{gbemphcolor}#1}}
\newcommand{\zd}[1]{%定义补充信息
{\small\youyuan\color{gbemphcolor}#1}} %orange,brown,purple,teal,gbblue,olive,cyan
\newcommand{\ZD}[1]{%定义补充信息
{\small\youyuan\color{gbemphcolor}#1}}


\newenvironment*{marglist}
{\list{}{\setlength{\topsep}{0pt}
\setlength{\partopsep}{0pt}
\setlength{\itemsep}{0pt}
\setlength{\parsep}{0pt}
\setlength{\leftmargin}{0pt}%
\setlength{\itemindent}{0pt}%
\renewcommand*{\makelabel}[1]{\hss\llap{\footnotesize\color{orange}\bfseries##1}}}}
{\endlist}

\makeatletter
\newcommand{\updateinfo}[2][\@empty]{%
\par\small\addvspace{2ex plus 1ex}%
\noindent{\color{gbemphcolor}\rule{\linewidth}{2pt}}
\vskip -\parskip
\ifx\@empty#1 \begin{marglist} \item #2\end{marglist}
\else \begin{marglist} \item[#1] #2\end{marglist} \fi}
\makeatother

%\usepackage{filecontents}



