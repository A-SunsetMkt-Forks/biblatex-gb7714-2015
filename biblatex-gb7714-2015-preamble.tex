\usepackage{expl3,etoolbox,ifthen,xstring}
\usepackage{ctex}
\let\kaiti=\kaishu
\usepackage{xeCJKfntef}
\usepackage[paperwidth=210mm,paperheight=290mm,left=20mm,right=20mm,top=25mm, bottom=15mm]{geometry}%定义版面
\usepackage[colorlinks=true,%
pdfstartview=FitH,linkcolor=blue,anchorcolor=violet,citecolor=magenta]{hyperref}  %书签功能,选项去掉链接红色方框
\usepackage{titleref} %标题引用
%linkcolor=black,linkcolor=green,blue,red,cyan, magenta,
%yellow, black, gray,white, darkgray, lightgray, brown,
%lime, olive, orange, red,purple, teal, violet.
%CJKbookmarks,bookmarksnumbered=true,

\setlength{\bibitemsep}{2pt}
\defbibheading{bibliography}[\bibname]{%
%\phantomsection%解决链接指引出错的问题,相当于加入了一个引导点
%\addcontentsline{toc}{subsection}{#1}
	\centering\subsubsection*{#1}}%

%自定义下划红线和背景颜色
\usepackage{ulem}
\newcommand\yellowback{\bgroup\markoverwith
{\textcolor{yellow}{\rule[-0.5ex]{2pt}{2.5ex}}}\ULon}
\newcommand\reduline{\bgroup\markoverwith
{\textcolor{red}{\rule[-0.5ex]{2pt}{0.4pt}}}\ULon}

\usepackage{subfigure}
\usepackage[subfigure]{tocloft} %注意其与titletoc共用时分页会有问题
\usepackage{ccaption}
\captiondelim{. } %图序图题中间的间隔符号
\captionnamefont{\zihao{-5}\heiti} %图序样式
\captiontitlefont{\zihao{-5}} %图题样式
\captionwidth{0.8\linewidth} %标题宽度
\changecaptionwidth
\captionstyle{\centering} %\captionstyle{<style>} style are: \centering, \raggedleft or \raggedright
%\precaption{\rule{\linewidth}{0.4pt}\par}
%\postcaption{\vspace{-1cm}}
\setlength{\belowcaptionskip}{2pt} %设置caption上下间距
\setlength{\abovecaptionskip}{0pt}
%\setlength{\abovelegendskip}{0pt}  %设置legend上下间距
%\setlength{\belowlegendskip}{0pt}
%新的浮动体设置
\newcommand{\listegcodename}{示~~ 例}%listegcodename,新环境目录的标题
\newcommand{\egcodename}{示例}%egcodename,新环境标题的图序
\newfloatlist{egcode}{loc}{\listegcodename}{\egcodename}%loc,写入条目的文件的扩展名
\newfixedcaption{\codecaption}{egcode}%egcode,环境名
%新的浮动体目录命令
%可以设置\cftbeforeZtitleskip,\cftafterZtitleskip长度,比如:
\setlength{\cftbeforeloctitleskip}{\baselineskip}
\setlength{\cftafterloctitleskip}{0.5\baselineskip}
%可以设置\cftZtitlefont字体样式比如
\renewcommand{\cftloctitlefont}{\hfill\Large\heiti}
\renewcommand{\cftafterloctitle}{\hfill}
\renewcommand{\cftegcodepresnum}{例}
\renewcommand{\cftegcodeaftersnum}{.}
\renewcommand{\cftegcodeaftersnumb}{~}
\cftsetindents{egcode}{0em}{3em}
\setlength{\cftbeforeegcodeskip}{1pt}%各条目的垂直间距
\renewcommand{\cftegcodepagefont}{\bfseries}
\setlength{\cftbeforetoctitleskip}{\baselineskip}
\setlength{\cftaftertoctitleskip}{0.5\baselineskip}
%目录命令
\renewcommand{\cfttoctitlefont}{\hfill\Large\heiti}
\renewcommand{\cftaftertoctitle}{\hfill\hspace{0.1cm}}
\renewcommand\contentsname{目~~ 录}
\renewcommand{\cftsecleader}{\leaders\hbox to 1em{\hss.\hss}\hfill}
\setlength{\cftbeforesecskip}{2mm}
\renewcommand{\cftsecfont}{\songti}

\usepackage{listings}
\usepackage{color,xcolor}
\colorlet{examplefill}{yellow!80!black}
\definecolor{graphicbackground}{rgb}{0.96,0.96,0.8}
\definecolor{codebackground}{rgb}{0.9,0.9,1}

\newlength{\outleftdis}
\newcommand{\codetopline}[1]{\settowidth{\outleftdis}{\footnotesize{\kaishu{#1}}~~~}%
\ifodd\thepage%判断奇数页,画一条有左标签的线
{\noindent\hspace*{-\outleftdis}\colorbox{yellow!50}{\color{blue}\footnotesize{\kaishu{#1}}~}\color{green!50!black}\rule[0.1\baselineskip]{\linewidth}{0.4pt}}%
\else%否则为偶数页,画一条有右标签的线
{%\noindent\makebox{\color{green!50!black}\rule[0.1\baselineskip]{\linewidth}{0.4pt}\colorbox{yellow}{\color{blue}~\footnotesize{#1}}}
\noindent\color{green!50!black}\rule[0.1\baselineskip]{\linewidth}{0.4pt}\colorbox{yellow!50}{\color{blue}~\footnotesize{\kaishu{#1}}}%
}%
\fi%
}
\newcommand{\codebottomline}
{\noindent\makebox{\color{green!50!black}\rule[1pt]{\linewidth}{0.4pt}}}

\lstnewenvironment{codetex}[2]%
{\begin{center}\end{center}\centering\setlength{\abovecaptionskip}{1mm}\setlength{\belowcaptionskip}{-3mm}%
\vspace{-2.0\baselineskip}
\codecaption{#1}\label{#2}%\nopagebreak
%\vspace{-1.0\baselineskip}
\codetopline{代码}%
\vspace{-0.5\baselineskip}
\lstset{% general command to set parameter(s)
%name=#1,
%label=#2,
%caption=\lstname,
linewidth=\linewidth,
breaklines=true,
%showspaces=true,
extendedchars=false,
columns=fullflexible,%flexible,
%frame=tb,
frame=none,
fontadjust=true,
language=[LaTeX]TeX,
%backgroundcolor=\color{yellow}, %背景颜色
backgroundcolor=\color{blue!5},%codebackground
numbers=left,
numberstyle=\tiny\color{red},
basicstyle=\small, % print whole listing small,footnotesize
keywordstyle=\color{blue}\bfseries,%\underbar,
% underlined bold black keywords
identifierstyle=, % nothing happens
commentstyle=\color{green!50!black}, % white comments
stringstyle=\ttfamily, % typewriter type for strings
showstringspaces=false}% no special string spaces
%\renewcommand{\baselinestretch}{0.9} %加这句的话需要进行垂直空间位置调整
}
{\nopagebreak\vspace*{-\baselineskip}\codebottomline\\}

\lstnewenvironment{texlist}%
{\lstset{% general command to set parameter(s)
%name=#1,
%label=#2,
%caption=\lstname,
linewidth=\linewidth,
breaklines=true,
%showspaces=true,
extendedchars=false,
columns=fullflexible,%flexible,
%frame=tb,
fontadjust=true,
language=[LaTeX]TeX,
%backgroundcolor=\color{yellow}, %背景颜色
numbers=left,
numberstyle=\tiny\color{red},
basicstyle=\footnotesize, % print whole listing small
keywordstyle=\color{blue}\bfseries,%\underbar,
% underlined bold black keywords
identifierstyle=, % nothing happens
commentstyle=\color{green!50!black}, % white comments
stringstyle=\ttfamily, % typewriter type for strings
showstringspaces=false}% no special string spaces
}
{}

\usepackage[listings,theorems]{tcolorbox}

\newcounter{myprop}\def\themyprop{\arabic{myprop}}
%序号如果带章节的话可以改为比如:\thesection.\arabic{myprop}
\tcbmaketheorem{property}{方法}{theorem style=plain,fonttitle=\bfseries\upshape, %
fontupper=\slshape,boxrule=0mm,arc=0mm, %
coltitle=black,colback=green!25,colframe=blue!50,%
separator sign={\ $\blacktriangleright$},
description delimiters={}{},
terminator sign={\ }}{myprop}{pp}
%最后一个必须参数是prefix用来做label比如这里是pp:加上给出的标签名

\newtcbtheorem[]{refentry}{条目类型}
{separator sign={\ $\blacktriangleright$},theorem style=plain,fonttitle=\bfseries,
fontupper=\normalsize,boxrule=0mm,arc=0mm,
coltitle=green!35!black,colbacktitle=green!15!white,
colback=green!50!yellow!15!white,terminator sign={\ }}{rfeg}
%最后一个必须参数是prefix用来做label比如这里是eg:加上给出的标签名

\newcommand{\titleformanual}[1]{\def\biaotiudf{#1}}
\newcommand{\authorformanual}[1]{\def\zuozheudf{#1}}
\newcommand{\dateformanual}[1]{\def\riqiudf{#1}}
%\ifthenelse{\equal{\youwuudf}{\temp}}{true}{false}
\def\temp{}
\newcommand{\titleandauthor}{
\begin{center}
{\renewcommand{\thefootnote}{\fnsymbol{footnote}}\setlength{\baselineskip}{30pt}\heiti{\zihao{-2}{\biaotiudf}}\par}
%注意这里\par要放在花括号内才有效
\vspace*{0.3cm}
{\renewcommand{\thefootnote}{\arabic{footnote}}\kaishu{\zihao{4}{\zuozheudf}}\par}
\vspace*{0.2cm}
{\songti{\zihao{-4}{\riqiudf}}\par}
\end{center}
}

\renewcommand{\thefootnote}{\textcircled{\tiny\arabic{footnote}}}

%--------------列表环境---------------------------------------------
\usepackage[inline]{enumitem} %重设list环境
\setlist[enumerate]{label=\bfseries\textcolor{violet}{\arabic*.},topsep=2pt,partopsep=0pt,parsep=0pt,%
align=left,leftmargin=0em,itemsep=0.5em,labelwidth=0.1em,itemindent=2.6em}%label=$\triangleright$,itemindent=1em
\setlist[itemize]{topsep=2pt,partopsep=0pt,parsep=0pt,%
leftmargin=4.5em,itemindent=0em}
\setlist[description]{font=\bfseries\textcolor{violet},align=right,topsep=5pt,partopsep=0pt,parsep=0pt,%
itemsep=0pt,leftmargin=2em,itemindent=0em}%注意,font或format中的最后一个命令自动提取标签为其参数


\usepackage{longtable}

\newcommand{\pz}[1]{%定义pz为旁注命令
\marginpar[\flushright\youyuan\color{red}\footnotesize #1]{\youyuan\color{red}\small #1}}
\newcommand{\PZ}[1]{%定义pz为旁注命令
\marginpar[\flushright\youyuan\color{red}\footnotesize  #1]{\youyuan\color{red}\small #1}}
\newcommand{\qd}[1]{%定义qd为强调命令
\begin{quote}
  \fangsong\color{blue}#1
\end{quote}}
\newcommand{\QD}[1]{%定义qd为强调命令
\begin{quote}
  \fangsong\color{blue}#1
\end{quote}}
\newcommand{\bc}[1]{%定义补充信息
{\kaiti\color{teal}#1}} %orange,brown,purple,teal,violet,olive,cyan
\newcommand{\BC}[1]{%定义补充信息
{\kaiti\color{teal}#1}}


\usepackage{amssymb}
\newcommand{\HandRight}{$\bigstar$}
\newcommand{\zhongdian}[1]{\textcolor{red}{\HandRight\heiti#1}}

\newenvironment*{marglist}
{\list{}{\setlength{\topsep}{0pt}
\setlength{\partopsep}{0pt}
\setlength{\itemsep}{0pt}
\setlength{\parsep}{0pt}
\setlength{\leftmargin}{0pt}%
\setlength{\itemindent}{0pt}%
\renewcommand*{\makelabel}[1]{\hss\llap{\footnotesize\color{orange}\bfseries##1}}}}
{\endlist}

\makeatletter
\newcommand{\updateinfo}[2][\@empty]{%
\par\small\addvspace{2ex plus 1ex}%
\noindent{\color{orange}\rule{\linewidth}{2pt}}
\vskip -\parskip
\ifx\@empty#1 \begin{marglist} \item #2\end{marglist}
\else \begin{marglist} \item[#1] #2\end{marglist} \fi}
\makeatother

\usepackage{filecontents}
\begin{filecontents}{egspecialchar.bib}
@Inproceedings{ref-replace-char,
  Title                    = {Cognitive Radio and Cooperative Strategies for Power Saving in Multi-Standard Wireless Devices},
  Address                  = { Florence, Italy},
  Author                   = {Rodriguez, J. and P. Marques and A. Radwan and K. Moessner and R. Tafazolli and others},
  Booktitle                = {Future % Network & Mobile # Summit 2010},
  Date                     = {June 2010}
}

@Online{olnoauthorcn,
  Title                    = {如何在LaTeX写作中管理参考文献?},
  Date                     = {2016-08-12},
  Url                      = {http://www.latexstudio.net/archives/7131}
}

@Online{Allianceurlonly,
  Url                      = {www.wimedia.org}
}

@Online{olnoauthoren,
  Title                    = {Dublin Core metadata element set: version 1.1},
  Url                      = {http://dublincore.org},
  Urldate                  = {2014-06-11},
  Year                     = {2012-06-14}
}
\end{filecontents}


